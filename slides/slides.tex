\documentclass[aspectratio=169, table]{beamer}

\usepackage{listings}
\usepackage{tikz}
\usetikzlibrary{positioning, arrows.meta, fit}



\lstdefinestyle{RustStyle}{
	language=Java,
	morekeywords={println, Ok, async, fn, main, use, let, mut},
	basicstyle=\ttfamily\scriptsize,
	keywordstyle=\color{blue},
	commentstyle=\color{gray},
	stringstyle=\color{red},
	breaklines=true,
	showstringspaces=false,
	tabsize=2,
	captionpos=b,
	numbers=left,
	numberstyle=\tiny\color{gray},
	frame=lines,
	backgroundcolor=\color{lightgray!10},
	comment=[l]{//},
	morecomment=[s]{/*}{*/},
	commentstyle=\color{gray}\ttfamily,
	string=[s]{'}{'},
	morestring=[s]{"}{"},
	stringstyle=\color{teal}\ttfamily,
	%	showstringspaces=false
	literate=
	{\{}{{\textcolor{red}{\{}}}1
	{\}}{{\textcolor{red}{\}}}}1
	{:}{{\textcolor{red}{:}}}1
	{=}{{\textcolor{red}{=}}}1
	{.}{{\textcolor{red}{.}}}1
	{]}{{\textcolor{red}{]}}}1
	{[}{{\textcolor{red}{[}}}1
	{\#}{{\textcolor{red}{\#}}}1
	{;}{{\textcolor{red}{;}}}1
	{?}{{\textcolor{red}{?}}}1
	{!}{{\textcolor{red}{!}}}1
}

%\usepackage[beamertheme=./praditatheme]{Pradita}

\usetheme{Pradita}



\title{\Huge Pelatihan AI\\\vspace{10pt}untuk Guru-guru}
\subtitle{Pengabdian Kepada Masyarakat}
%\date[Serial]{Penggunaan Large Language Model untuk Pengajaran}
\author{\textbf{Alfa Yohannis}}
\begin{document}
	
	\frame{\titlepage}
	
	
	\begin{frame}[fragile]
		\frametitle{Materi 1}
		\vspace{20pt}
		\begin{columns}[t]
			\column{0.5\textwidth}
			\tableofcontents[sections={1-3}]
			
			\column{0.5\textwidth}
			\tableofcontents[sections={4-6}]
		\end{columns}
	\end{frame}


\section{Bab 1: Pengenalan ChatGPT}
\begin{frame}{\hfill}
	\centering
	\Huge{\textbf{Bab 1: Pengenalan ChatGPT}}
\end{frame}
	

		
		\begin{frame}[fragile]{Apa itu Large Language Model (LLM)?}
			\vspace{20pt}
			\begin{columns}[T]
				\column{0.5\textwidth}
				\small
				\textbf{Definisi dan Mekanisme:}
				\begin{itemize}
					\item LLM adalah AI yang memahami dan menghasilkan bahasa alami.
					\item Dilatih dari teks dalam jumlah besar (buku, artikel, web).
					\item Teks diubah jadi \textbf{token} (kata/suku kata).
					\item Diproses oleh \textbf{jaringan saraf bertingkat}.
					\item Memahami pola dan konteks antar-token.
				\end{itemize}
				
				\vspace{5pt}
				\textbf{Ukuran "Large":}
				\begin{itemize}
					\item Miliaran parameter untuk memahami konteks kompleks.
				\end{itemize}
				
				\column{0.5\textwidth}
				\small
				\textbf{Contoh LLM Populer:}
				\begin{itemize}
					\item \textbf{ChatGPT} (OpenAI) – percakapan, pendamping belajar.
					\item \textbf{Claude} (Anthropic) – fokus keamanan dan kejelasan.
					\item \textbf{Gemini} (Google) – integrasi di produk Google.
				\end{itemize}
				
				\textbf{Manfaat dalam Pendidikan:}
				\begin{itemize}
					\item Termasuk generative AI.
					\item Membantu guru menyiapkan materi dan umpan balik.
					\item Mendukung pembelajaran personal.
				\end{itemize}
			\end{columns}
		\end{frame}
	

\begin{frame}[fragile]{Cara Kerja LLM (Sederhana)}
	\vspace{10pt}
	
	% Diagram
	\begin{center}
		\tikzstyle{process} = [rectangle, rounded corners, minimum width=2.2cm, minimum height=1.1cm, text centered, align=center, draw=black, fill=blue!10, font=\bfseries]
		\tikzstyle{arrow} = [thick,->,>=stealth]
		
		\begin{tikzpicture}[node distance=.1cm, every node/.style={process}]
			\node (data) {Data\\Teks Besar};
			\node (training) [right of=data, xshift=2.6cm] {Pelatihan\\Model};
			\node (input) [right of=training, xshift=2.8cm] {Input\\Pengguna};
			\node (token) [right of=input, xshift=3cm] {Tokenisasi\\\& Analisis};
			\node (output) [right of=token, xshift=3.2cm] {Prediksi\\\& Output};
			
			\draw [arrow] (data) -- (training);
			\draw [arrow] (training) -- (input);
			\draw [arrow] (input) -- (token);
			\draw [arrow] (token) -- (output);
		\end{tikzpicture}
	\end{center}
	
	\vspace{12pt}
	
	% Penjelasan Ringkas
	\scriptsize
	\begin{itemize}
		\item \textbf{1. Data Teks Besar:} LLM belajar dari miliaran kata dari berbagai sumber.
		\item \textbf{2. Pelatihan Model:} Model belajar mengenali pola dan konteks kata.
		\item \textbf{3. Input Pengguna:} Teks masukan digunakan untuk meminta respons.
		\item \textbf{4. Tokenisasi dan Analisis:} Teks dipecah jadi token dan dianalisis secara kontekstual.
		\item \textbf{5. Prediksi dan Output:} Model menyusun jawaban dari prediksi token bertahap.
	\end{itemize}
	
\end{frame}


\begin{frame}{\hfill}
	\centering
	\Huge{\textbf{Demo ChatGPT}}
\end{frame}
	

\begin{frame}[fragile]{Studi Kasus Penggunaan LLM}
\vspace{20pt}
	LLM seperti ChatGPT membantu guru menyiapkan materi, memperkaya pembelajaran, dan menyesuaikan dengan kebutuhan siswa:
	
	\begin{itemize}
		\item \textbf{Materi dan Soal:}  
		\texttt{"Buat ringkasan dan soal tentang perubahan iklim untuk siswa SMA"}.
		
		\item \textbf{Penyederhanaan Teks:}  
		\texttt{"Sederhanakan artikel ini untuk siswa kelas 10"}.
		
		\item \textbf{Umpan Balik Otomatis:}  
		\texttt{"Berikan umpan balik atas tulisan ini"}.
		
		\item \textbf{Adaptasi Khusus:}  
		\texttt{"Ubah teks ini jadi versi dengan kalimat pendek"}.
		
		\item \textbf{Simulasi Diskusi:}  
		\texttt{"Tulis percakapan Soekarno dan Gandhi soal kemerdekaan"}.
	\end{itemize}
	
	\textit{Catatan:} Tetap diperlukan supervisi agar hasil relevan dan sesuai konteks.
\end{frame}


\begin{frame}[fragile]{Metode Penggunaan LLM}
	\vspace{20pt}
	\begin{enumerate}
		\item \textbf{Inisiasi Interaksi:}  
		Buka \url{https://chatgpt.com}.
		
		\item \textbf{Coba Prompt Berikut:}
		\begin{itemize}
			\item \texttt{"Buat rencana pelajaran 1 jam tentang fotosintesis."}
			\item \texttt{"Buat 5 soal pilihan ganda tentang revolusi industri."}
		\end{itemize}
		
		\item \textbf{Evaluasi Hasil:}  
		Tinjau apakah jawaban sesuai konteks pembelajaran dan lakukan perbaikan jika perlu.
		
		\item \textbf{Eksplorasi Mandiri:}  
		Coba prompt lain sesuai bidang atau kebutuhan. Ubah gaya atau format untuk melihat variasi hasil.
		
		\item \textbf{Simpan dan Dokumentasikan:}  
		Tangkap layar atau salin hasil yang relevan untuk referensi atau bahan ajar selanjutnya.
	\end{enumerate}
\end{frame}

\section{Latihan Praktik: Mengeksplorasi Prompt ChatGPT}

\begin{frame}[fragile]{Latihan Praktik: Mengeksplorasi Prompt}
	\vspace{20pt}
	\begin{itemize}
		\item \textbf{Tujuan:} Memahami bagaimana struktur prompt memengaruhi respons AI.
		
		\item \textbf{Tugas:} Coba prompt berikut di ChatGPT atau LLM lainnya:
		\begin{quote}
					\vspace{5pt}
			\centering
			\texttt{"Buat rencana pembelajaran 30 menit untuk pelajaran biologi SMA tentang inflasi pada ekonomi."}
		\end{quote}
		
		\item \textbf{Tantangan:} Modifikasi prompt untuk:
		\begin{itemize}
			\item Mata pelajaran lain (misalnya: matematika, sastra)
			\item Tingkat kelas berbeda (SMP atau SD)
			\item Format alternatif (misalnya: kerja kelompok, flipped classroom)
			\item Eksperimen dengan kondisi pembelajaran lainnya
		\end{itemize}
	\end{itemize}
\end{frame}

\section{Bab 2: Rencana Pembelajaran dan Lembar Kerja}
\begin{frame}{\hfill}
	\centering
	\Huge{\textbf{Bab 2: Rencana Pembelajaran dan Lembar Kerja}}
\end{frame}

\section{Mengapa Perencanaan Pembelajaran yang Efisien Itu Penting}

\begin{frame}[fragile]{Efisiensi dalam Perencanaan Itu Penting}
	\vspace{20pt}
	\small
	\begin{itemize}
		\item Perencanaan pembelajaran adalah fondasi pengajaran yang efektif dan terstruktur.
		
		\item Guru sering menghadapi keterbatasan waktu, sumber daya, dan kebutuhan siswa yang beragam.
		
		\item LLM membantu mengotomatisasi penyusunan:
		\begin{itemize}
			\item Tujuan pembelajaran
			\item Strategi pengajaran
			\item Soal evaluasi
		\end{itemize}
		
		\item Guru tidak perlu mulai dari nol — cukup berikan deskripsi topik atau tujuan.
		
		\item Mendukung diferensiasi: materi dapat disesuaikan untuk berbagai tingkat kemampuan dan gaya belajar.
		
		\item LLM memungkinkan pembuatan versi materi adaptif untuk kelas heterogen.
		
		\item Efisiensi memberi banyak waktu untuk interaksi bermakna dan pembimbingan individu.
	\end{itemize}
\end{frame}

\begin{frame}[fragile]{Menyusun Rencana Pembelajaran}
	\vspace{20pt}
	\small
	\begin{columns}[T]
		\column{0.5\textwidth}
		\textbf{Peran LLM:}
		\begin{itemize}
			\item Menyusun rencana secara cepat dan fleksibel
			\item Cukup beri topik, jenjang, dan tujuan
			\item Hasil mencakup:
			\begin{itemize}
				\item Tujuan pembelajaran
				\item Kegiatan (awal, inti, penutup)
				\item Media dan evaluasi
			\end{itemize}
		\end{itemize}
		
		\textbf{Contoh Prompt:}
		\begin{quote}
			\centering
			\texttt{"Buatkan rencana pembelajaran 1 jam untuk SMA kelas X tentang sistem pernapasan manusia."}
		\end{quote}
		
		\column{0.5\textwidth}
		\textbf{Hasil Umum:}
		\begin{itemize}
			\item \textbf{Tujuan:} siswa menjelaskan proses pernapasan dan organ terkait
			\item \textbf{Kegiatan:}
			\begin{itemize}
				\item Tanya jawab dan video
				\item Diskusi kelompok + presentasi
				\item Refleksi + kuis 5 soal
			\end{itemize}
		\end{itemize}
		
		\textbf{Penyesuaian:}
		\begin{itemize}
			\item Kebutuhan khusus (misal: narasi audio)
			\item Versi bilingual / Bahasa Inggris
			\item Gaya ajar: PBL, tematik, dll.
			\item Kurikulum dan konteks kelas
		\end{itemize}
		
		
	\end{columns}
\end{frame}


\begin{frame}[fragile]{Membuat Soal dan Kuis}
	\vspace{20pt}
	\small
	\begin{columns}[T]
		\column{0.5\textwidth}
		\textbf{Mengapa Penting:}
		\begin{itemize}
			\item Membuat soal memakan waktu
			\item Harus relevan dan terstruktur
			\item LLM bantu percepat dan variasikan soal
		\end{itemize}
		
		\textbf{Jenis Soal:}
		\begin{itemize}
			\item Pilihan Ganda, Isian Singkat, Uraian Pendek, Benar/Salah, Menjodohkan
		\end{itemize}
		
		\textbf{Contoh Prompt:}
		\begin{quote}
			\centering
			\texttt{"Buatkan 5 soal pilihan ganda tentang revolusi industri kelas XI IPS"}
		\end{quote}
		
		\column{0.5\textwidth}
		\textbf{Hasil LLM:}
		\begin{itemize}
			\item Soal dan pilihan A–D
			\item Jawaban benar + penjelasan
		\end{itemize}
		
		\textbf{Penggunaan:}
		\begin{itemize}
			\item Cetak atau unggah ke platform digital
			\item Latihan harian, remedial, tugas kelompok
		\end{itemize}
		
		\textbf{Penyesuaian:}
		\begin{itemize}
			\item Ubah gaya, tingkat, atau konteks
			\item Minta variasi soal HOTS/kontekstual
		\end{itemize}
		
		\textbf{Catatan:} Tetap periksa hasil LLM sebelum digunakan

	\end{columns}
\end{frame}




\begin{frame}[fragile]{Latihan Membuat Materi dan Kuis}
	\vspace{20pt}
	\small
	\begin{itemize}
		\item \textbf{Tujuan:} Membuat lembar kerja dan kuis yang sesuai dengan topik serta kebutuhan peserta didik.
		
		\item \textbf{Tugas:} Gunakan LLM untuk menghasilkan:
		\begin{itemize}
			\item Rencana pembelajaran berdurasi 1 jam untuk topik pilihan.
			\item Lima soal pilihan ganda beserta jawabannya.
			\item Lembar kerja yang disesuaikan untuk siswa dengan kebutuhan belajar khusus, seperti:
			\begin{itemize}
				\item Teks ringan dan singkat
				\item Berbasis gambar
				\item Bertingkat kesulitan
			\end{itemize}
		\end{itemize}
	\end{itemize}
\end{frame}

\section{Bab 3: Memberi Masukan dan Rubrik Penilaian}
\begin{frame}{\hfill}
	\centering
	\Huge{\textbf{Bab 3: Memberi Masukan dan Penilaian}}
\end{frame}


\begin{frame}[fragile]{Menyusun Umpan Balik}
	\vspace{20pt}
	\small
	\begin{columns}[T]
		\column{0.5\textwidth}
		\textbf{Mengapa Penting:}
		\begin{itemize}
			\item Umpan balik bantu tingkatkan pemahaman dan tantangan jika kelas besar. LLM bantu buat komentar cepat dan relevan.
		\end{itemize}
		
		\textbf{Tugas yang Cocok:}
		\begin{itemize}
			\item Esai dan refleksi, Laporan eksperimen, Jawaban uraian
		\end{itemize}
		
		\textbf{Prompt Utama:}
		\begin{quote}
			\centering
			\texttt{"Berikan umpan balik konstruktif untuk esai berikut. Tinjau isi, struktur argumen, dan bahasa: [teks esai]."}
		\end{quote}
		
		\column{0.5\textwidth}
		\textbf{Hasil yang Diharapkan:}
		\begin{itemize}
			\item Apresiasi terhadap kekuatan isi
			\item Koreksi dan saran konkret
			\item Nada sesuai konteks siswa
		\end{itemize}
		
		\textbf{Contoh Gaya Ramah:}
		\begin{quote}
			\centering
			\texttt{"Tulis umpan balik ramah dan memotivasi untuk siswa SMP."}
		\end{quote}
		
		\textbf{Tips Efektif:}
		\begin{itemize}
			\item Gunakan prompt yang jelas dan terarah
			\item Tinjau dan sesuaikan hasilnya
			\item Jadikan dasar diskusi atau refleksi siswa
		\end{itemize}
	\end{columns}
\end{frame}


\begin{frame}[fragile]{Membuat Rubrik Penilaian}
	\vspace{20pt}
	\small
	\begin{columns}[T]
		\column{0.5\textwidth}
		\textbf{Mengapa Rubrik Penting:}
		\begin{itemize}
			\item Bantu penilaian objektif dan konsisten. Siswa tahu harapan dan kriteria. Proses manual memakan waktu.
		\end{itemize}
		
		\textbf{Aspek Umum Rubrik:}
		\begin{itemize}
			\item Isi dan relevansi konten, Struktur tulisan, Tata bahasa, Orisinalitas gagasan
		\end{itemize}
		
		\textbf{Prompt Utama:}
		\begin{quote}
			\centering
			\texttt{"Buatkan rubrik esai untuk siswa SMA.\\Aspek: isi, struktur, bahasa, orisinalitas.\\Format 4 level."}
		\end{quote}
		
		\column{0.5\textwidth}
		\textbf{Contoh Hasil (Ringkas):}
		\begin{itemize}
			\item \textbf{Isi:} Lengkap vs tidak sesuai topik
			\item \textbf{Struktur:} Rapi vs tidak jelas
			\item \textbf{Bahasa:} Baku vs banyak kesalahan
			\item \textbf{Orisinalitas:} Orisinal vs menyalin
		\end{itemize}
		
		\textbf{Prompt Alternatif SD:}
		\begin{quote}
			\centering
			\texttt{"Buat rubrik presentasi kelas 5 SD.\\Aspek: kejelasan, gambar, kerja sama."}
		\end{quote}
		
		\textbf{Manfaat:} Cepat dan mudah disesuaikan, Tetap bisa disunting guru
	\end{columns}
\end{frame}


\begin{frame}[fragile]{Latihan Umpan Balik dan Rubrik Penilaian}
	\vspace{20pt}
	\small
	\begin{itemize}
		\item \textbf{Tujuan:} Menyusun umpan balik dan rubrik yang sesuai dengan karakteristik tugas dan capaian pembelajaran.
		
		\item \textbf{Tugas:}
		\begin{itemize}
			\item Evaluasi esai siswa dengan LLM dan berikan umpan balik otomatis.
			\item Buat rubrik penilaian untuk tugas presentasi atau proyek kelompok.
			\item Susun 5 pertanyaan refleksi untuk self-assessment siswa.
		\end{itemize}
	\end{itemize}
\end{frame}


\section{Bab 4: Riset and Merangkum}
\begin{frame}{\hfill}
	\centering
	\Huge{\textbf{Bab 4: Riset and Merangkum}}
\end{frame}

\begin{frame}[fragile]{Mencari dan Membuat Sumber Belajar Baru}
	\vspace{20pt}
	LLM seperti ChatGPT dapat membantu guru menemukan dan menyusun sumber belajar secara cepat dan relevan.
	
	\textbf{Contoh Prompt untuk Pencarian Sumber:}
	\begin{quote}
		\centering
		\texttt{"Rekomendasikan 5 sumber belajar daring untuk topik fotosintesis tingkat SMP."}
	\end{quote}
	Model akan menyajikan daftar sumber dengan penjelasan singkat dan tingkat kesesuaian dengan siswa.
	
	\textbf{Contoh Prompt untuk Pembuatan Materi:}
	\begin{quote}
		\centering
		\texttt{"Buatkan teks bacaan informatif 150 kata untuk siswa kelas 4 SD tentang siklus air, sertakan pertanyaan reflektif di akhir."}
	\end{quote}
	LLM dapat menghasilkan penjelasan konsep, narasi, teks bacaan, hingga eksperimen mini.
\end{frame}

\begin{frame}[fragile]{Menyusun Laporan Otomatis dengan LLM}
	\vspace{20pt}
	LLM seperti ChatGPT dapat menyusun laporan berdasarkan poin penting atau catatan observasi secara otomatis dan terstruktur.
	
	\vspace{5pt}
	\textbf{Contoh Prompt untuk Laporan Diskusi:}
	\begin{quote}
		\centering
		\texttt{"Susun laporan hasil diskusi kelompok tentang perubahan iklim berdasarkan poin berikut: siswa memahami penyebab, dampak, dan solusi. Sertakan pendahuluan, isi utama, dan kesimpulan."}
	\end{quote}

		\vspace{5pt}
	\textbf{Contoh Prompt untuk Laporan Observasi Siswa:}
	\begin{quote}
		\centering
		\texttt{"Buat laporan observasi untuk siswa bernama Adi. Catatan: aktif saat diskusi, menyampaikan pendapat dengan percaya diri, mulai berani bertanya, kadang kurang fokus saat penjelasan guru."}
	\end{quote}
\end{frame}

\begin{frame}[fragile]{Menggunakan LLM untuk Meringkas}
	\vspace{20pt}
	
	LLM seperti ChatGPT dapat membantu menyederhanakan artikel panjang menjadi ringkasan yang ringkas, mudah dipahami, dan relevan dengan kebutuhan belajar.
	
			\vspace{5pt}
	\textbf{Contoh Prompt Ringkasan:}
	\begin{quote}
		\centering
		\texttt{"Ringkas artikel berikut menjadi maksimal 5 kalimat yang mudah dipahami siswa SMA: [salin isi artikel]."}
	\end{quote}
	
			\vspace{5pt}
	\textbf{Variasi Ringkasan:} Tematik (fokus pada topik tertentu), Disesuaikan jenjang (SD, SMP, SMA), Format poin-poin
	
			\vspace{5pt}
	\textbf{Contoh Hasil:}
	\begin{quote}
		\centering
		\texttt{"Krisis air bersih memburuk akibat populasi dan iklim. Negara kesulitan memenuhi kebutuhan air. Solusi seperti desalinasi dikembangkan. Akses air jadi target SDGs. Perlu kesadaran masyarakat."}
	\end{quote}
\end{frame}

\begin{frame}[fragile]{Latihan Menyusun Informasi Pembelajaran }
	\vspace{20pt}
	
	\textbf{Tujuan:}  
	Menggunakan LLM untuk menyederhanakan dan menyusun ulang informasi pendidikan agar lebih mudah digunakan dalam konteks pembelajaran.
	
	\textbf{Tugas:}
	\begin{itemize}
		\item Ringkas artikel berita atau ilmiah menjadi 5–7 kalimat.
		\item Buat materi pembelajaran topik “ekosistem” untuk siswa SMP.
		\item Susun laporan otomatis dari teks hasil wawancara atau catatan observasi siswa.
	\end{itemize}
	
	Gunakan ChatGPT atau LLM lain untuk mengeksplorasi efisiensi dan fleksibilitas dalam menyusun ulang konten pendidikan.
\end{frame}


\section{Bab 5: Tugas Administratif and Komunikasi}
\begin{frame}{\hfill}
	\centering
	\Huge{\textbf{Bab 5: Tugas Administratif and Komunikasi}}
\end{frame}	

\begin{frame}[fragile]{Menyusun Email dan Surat}
	\vspace{20pt}
	
	Komunikasi tertulis dengan orang tua seperti undangan, laporan, atau klarifikasi dapat dibantu dengan LLM secara cepat dan konsisten.
	
	\vspace{10pt}
	\textbf{Contoh Prompt:}
	\begin{quote}
		\centering
		\texttt{"Tulis email resmi kepada orang tua siswa kelas 7 untuk mengundang mereka menghadiri pertemuan wali kelas pada hari Jumat, pukul 15.00 di aula."}
	\end{quote}
	
	\vspace{10pt}
	\textbf{Manfaat:} Efisiensi waktu, konsistensi format, dan kemudahan kustomisasi untuk tiap siswa atau kelas.
\end{frame}


\begin{frame}[fragile]{Menyusun Surat dan Dokumen Sekolah}
	\vspace{20pt}
	LLM seperti ChatGPT membantu menyusun surat dan dokumen resmi secara cepat dan konsisten. Cukup masukkan instruksi singkat, hasilnya siap diedit sesuai konteks.
	
	\vspace{10pt}
	\textbf{Contoh penggunaan:}  
	surat tugas; undangan rapat; notulensi; pengumuman siswa/orang tua; surat izin atau kerja sama.
	
	\vspace{10pt}
	\textbf{Manfaat:}  
	hemat waktu; format profesional; mudah disesuaikan untuk banyak siswa atau individu.
	
	\vspace{10pt}
	\textbf{Contoh Prompt:}
	\begin{quote}
		\centering
		\texttt{"Buat surat tugas untuk Ibu Ratna sebagai pendamping lomba OSN di SMAN 5, 22–24 Mei 2025. Sertakan kop surat dan format resmi."}
	\end{quote}
\end{frame}

\begin{frame}[fragile]{Laporan Akademik dan Nonakademik}
	\vspace{20pt}
	LLM seperti ChatGPT membantu menyusun laporan secara cepat dan konsisten, hanya dari poin-poin atau jurnal guru.
	
	\vspace{10pt}
	\textbf{Jenis laporan:}  
	laporan belajar, kegiatan kelas, refleksi siswa, perkembangan karakter.
	
	\vspace{10pt}
	\textbf{Contoh prompt:}
	\begin{quote}
		\centering
		\texttt{"Susun laporan hasil belajar siswa bernama Andi berdasarkan catatan: memahami materi, aktif bertanya, nilai stabil, kerja kelompok perlu ditingkatkan."}
	\end{quote}
	
	\vspace{10pt}
	\textbf{Manfaat:}  
	menghemat waktu, struktur konsisten, variasi redaksi, kualitas refleksi meningkat.
\end{frame}

\begin{frame}[fragile]{Merangkum Catatan dan Observasi Harian}
	\vspace{20pt}
	LLM membantu merapikan catatan harian dan observasi guru menjadi narasi reflektif atau notulensi yang siap dokumentasi.
	
	\vspace{10pt}
	\textbf{Dapat diringkas:}  
	observasi siswa, refleksi guru, diskusi kelompok, poin rapat, tanggapan siswa.
	
	\vspace{10pt}
	\textbf{Contoh prompt:}
	\begin{quote}
		\centering
		\texttt{"Rangkum observasi berikut menjadi paragraf reflektif: siswa aktif bertanya, kurang fokus di awal, tertarik saat eksperimen..."}
	\end{quote}
	
	\vspace{10pt}
	\textbf{Manfaat:}  
	hemat waktu, hasil rapi, dukung refleksi profesional, bantu dokumentasi berkelanjutan.
\end{frame}

\begin{frame}[fragile]{Latihan Komunikasi dan Dokumentasi}
	\vspace{20pt}
	\textbf{Tujuan:}  
	Membuat dokumen komunikasi dan laporan secara cepat dan tepat dengan bantuan LLM.
	
	\vspace{10pt}
	\textbf{Tugas:}
	\begin{itemize}
		\item Tulis surat pemberitahuan kegiatan kelas untuk orang tua.
		\item Susun laporan perkembangan siswa dari poin observasi.
		\item Buat notulensi rapat berdasarkan daftar topik diskusi.
	\end{itemize}
\end{frame}

\section{Bab 6: Etika}
\begin{frame}{\hfill}
	\centering
	\Huge{\textbf{Bab 6: Etika}}
\end{frame}	
	
	
\begin{frame}[fragile]{Etika Penting dalam Penggunaan AI}
	\vspace{10pt}
	\begin{columns}[T]
		\column{0.5\textwidth}
		\textbf{AI dan Risiko Etis} \\
		AI bantu efisiensi pembelajaran, tetapi tanpa etika dapat berisiko.
		
		\vspace{5pt}
		\textbf{Risiko Utama:}
		\begin{itemize}
			\item Bias – stereotip dari data
			\item Plagiarisme – tugas tanpa belajar
			\item Misinformasi – jawaban salah
			\item Ketergantungan – kurang kritis
			\item Privasi – bocor data siswa
		\end{itemize}
		
		\column{0.5\textwidth}
		\textbf{Peran Guru:}
		\begin{itemize}
			\item Tinjau hasil AI
			\item Sesuaikan konteks siswa
			\item Gunakan AI sebagai alat bantu
		\end{itemize}
		
		\vspace{5pt}
		\textbf{Tanggung Jawab Etis:}
		\begin{itemize}
			\item Nyatakan jika dibantu AI
			\item Dorong siswa tetap kritis
			\item Sunting hasil, jaga privasi
		\end{itemize}
	\end{columns}
\end{frame}


\begin{frame}[fragile]{Bias dalam AI dan Dampaknya}
	\vspace{20pt}
	\begin{columns}[t]
		\column{0.6\textwidth}
		\textbf{Apa itu Bias?} \\
		AI belajar dari data internet yang tak selalu netral.
		
		\vspace{4pt}
		\textbf{Jenis Bias:}
		\begin{itemize}
			\item Budaya dominan
			\item Stereotip gender
			\item Fokus bahasa Inggris
			\item Abaikan konteks lokal
		\end{itemize}
		
		\vspace{4pt}
		\textbf{Dampak:}
		\begin{itemize}
			\item Figur Barat lebih sering muncul
			\item Cerita atau soal kurang relevan
			\item Minoritas tidak terwakili
		\end{itemize}
		
		\column{0.4\textwidth}
		\textbf{Risiko di Pembelajaran:}
		\begin{itemize}
			\item Kelas tidak inklusif
			\item Stereotip diperkuat
			\item Hilang nilai lokal
		\end{itemize}
		
		\vspace{3pt}
		\textbf{Peran Guru:}
		\begin{itemize}
			\item Tinjau hasil AI
			\item Tambahkan konteks lokal
			\item Bahas bias bersama siswa
		\end{itemize}
		
		\vspace{3pt}
		\textbf{Contoh Prompt Korektif:}
		\texttt{"Sebutkan 5 tokoh perempuan Asia Tenggara di bidang pendidikan."}
	\end{columns}
\end{frame}

	
	\begin{frame}[fragile]{Risiko Misinformasi dari AI}
		\vspace{6pt}
		\begin{columns}[T]
			\column{0.48\textwidth}
			\textbf{Mengapa Misinformasi Terjadi?}
			\begin{itemize}
				\item AI prediksi teks, bukan verifikasi fakta
				\item Data pelatihan bisa usang
				\item Jawaban terlihat meyakinkan
				\item Konteks bisa disalahpahami
			\end{itemize}
			
			\textbf{Contoh Kesalahan Umum:}
			\begin{itemize}
				\item Tanggal sejarah tidak tepat
				\item Statistik palsu atau tidak ada sumber
				\item Kutipan dari buku yang fiktif
			\end{itemize}
			
			\column{0.48\textwidth}
			\textbf{Dampak pada Pendidikan:}
			\begin{itemize}
				\item Siswa percaya info salah
				\item Guru pakai materi keliru
				\item Kredibilitas proses belajar menurun
			\end{itemize}
			
			\textbf{Strategi Mengurangi Risiko:}
			\begin{itemize}
				\item Verifikasi jawaban AI secara mandiri
				\item Gunakan prompt dengan batasan tahun
				\item Ajarkan siswa cek ulang ke sumber
			\end{itemize}
		\end{columns}
	\end{frame}
	
	\begin{frame}[fragile]{Ketergantungan Berlebih terhadap AI}
		\vspace{15pt}
		\begin{columns}[T]
			\column{0.48\textwidth}
			\textbf{Apa yang Terjadi?}
			\begin{itemize}
				\item AI bantu banyak tugas
				\item Tapi bisa timbul ketergantungan
				\item Guru dan siswa pasif berpikir
			\end{itemize}
			
			\textbf{Contoh Ketergantungan:}
			\begin{itemize}
				\item Tugas diserahkan ke AI
				\item Guru salin materi dari AI
				\item Siswa malas berdiskusi
				\item Pembelajaran jadi generik
			\end{itemize}
			
			\column{0.48\textwidth}
			\textbf{Dampaknya:}
			\begin{itemize}
				\item Kemampuan kritis menurun
				\item Kreativitas dan orisinalitas hilang
				\item Tidak kontekstual, jadi kurang bermakna
				\item Kesenjangan makin lebar
			\end{itemize}
			
			\textbf{Solusi:}
			\begin{itemize}
				\item AI sebagai alat bantu
				\item Ajak refleksi atas jawabannya
				\item Dorong ekspresi dan gagasan asli
				\item Latih metakognisi dalam penggunaan
			\end{itemize}
		\end{columns}
	\end{frame}
	
	\begin{frame}[fragile]{Plagiarisme dan Keaslian Karya}
		\vspace{15pt}
		\begin{columns}[T]
			\column{0.48\textwidth}
			\textbf{Plagiarisme di Era AI}
			\begin{itemize}
				\item Kirim hasil AI mentah
				\item Jawab soal tanpa paham
				\item Sisipkan teks tanpa atribusi
			\end{itemize}
			
			\textbf{Tantangan Utama}
			\begin{itemize}
				\item Sulit dibedakan secara kasat mata
				\item Tidak terdeteksi alat cek umum
			\end{itemize}
			
			\textbf{Mengapa Keaslian Penting?}
			\begin{itemize}
				\item Latih tanggung jawab intelektual
				\item Fokus pada proses, bukan hasil
			\end{itemize}
			
			\column{0.48\textwidth}
			\textbf{Cara Mendidik Siswa}
			\begin{itemize}
				\item Jelaskan batas bantu vs salin
				\item Tambahkan refleksi pada tugas
				\item Tunjukkan contoh prompt etis
			\end{itemize}
			
			\textbf{Peran Guru dan Sekolah}
			\begin{itemize}
				\item Buat kebijakan penggunaan AI
				\item Bahas etika digital di kelas
				\item Nilai proses, bukan hanya hasil
			\end{itemize}
			
			\textbf{Kesimpulan}
			\begin{itemize}
				\item AI bantu, bukan pengganti
				\item Keaslian bukti keterlibatan belajar
			\end{itemize}
		\end{columns}
	\end{frame}
	
	\begin{frame}[fragile]{Privasi dan Keamanan Data}
		\vspace{15pt}
		\begin{columns}[T]
			\column{0.48\textwidth}
			\textbf{Mengapa Privasi Penting?}
			\begin{itemize}
				\item Identitas dan catatan siswa
				\item Data sensitif atau internal
			\end{itemize}
			
			\textbf{Kesalahan Umum}
			\begin{itemize}
				\item Masukkan data pribadi ke AI
				\item Kirim transkrip konseling online
			\end{itemize}
			
			\textbf{Risiko yang Timbul}
			\begin{itemize}
				\item Kebocoran data pribadi
				\item Pelanggaran hukum data
			\end{itemize}
			
			\column{0.48\textwidth}
			\textbf{Prinsip Aman Menggunakan AI}
			\begin{itemize}
				\item Hindari data lengkap
				\item Gunakan nama samaran
				\item Pilih AI tanpa simpan data
			\end{itemize}
			
			\textbf{Peran Sekolah dan Guru}
			\begin{itemize}
				\item Latih etika digital
				\item Buat kebijakan AI
			\end{itemize}
			
			\textbf{Kesimpulan}
			\begin{itemize}
				\item AI bantu, bukan simpan data
				\item Privasi tanggung jawab bersama
			\end{itemize}
		\end{columns}
	\end{frame}
	
	
	\begin{frame}[fragile]{Aman dan Bertanggung Jawab}
		\vspace{15pt}
		\begin{columns}[T]
			\column{0.48\textwidth}
			\textbf{Prinsip Penggunaan AI}
			\begin{itemize}
				\item Transparan jika pakai AI
				\item Evaluasi hasil, jangan terima mentah
				\item Bandingkan dengan sumber resmi
				\item Hindari masukkan data pribadi
				\item Jaga orisinalitas karya
			\end{itemize}
			
			\textbf{Peran Guru}
			\begin{itemize}
				\item Beri konteks fungsi AI
				\item Nilai proses, bukan hasil akhir
				\item Fasilitasi diskusi etika teknologi
			\end{itemize}
			
			\column{0.48\textwidth}
			\textbf{Aktivitas Etis AI}
			\begin{itemize}
				\item Bandingkan teks AI \& manusia
				\item Kombinasikan AI dan refleksi pribadi
				\item Ajak siswa buat pedoman sendiri
			\end{itemize}
			
			\textbf{Checklist Siswa}
			\begin{itemize}
				\item Apakah saya paham hasil AI?
				\item Sudah cek sumber lain?
				\item Apakah saya menyebutkan AI?
				\item Apakah saya refleksi sendiri?
			\end{itemize}
		\end{columns}
	\end{frame}
	
\begin{frame}[fragile]{Latihan Analisis Risiko dan Refleksi Etis (1)}
	\vspace{10pt}
	\begin{columns}[T]
		\column{0.48\textwidth}
		\textbf{Tujuan Latihan:}
		\begin{itemize}
			\item Identifikasi risiko AI di kelas
			\item Analisis studi kasus nyata/simulasi
			\item Pertimbangkan aspek etika \& keamanan
			\item Rancang pedoman AI yang bijak
		\end{itemize}
		
		
		
		\column{0.48\textwidth}
		\textbf{Studi Kasus:}
		\begin{itemize}
			\item \textbf{Kasus A:} Siswa serahkan esai hasil kerja dari AI
			\item \textbf{Kasus B:} Guru memberi nilai suatu esai berdasarkan nilai yang diberikan oleh AI
		\end{itemize}
		\vspace{10pt}
		\textbf{Diskusi Kelompok:}
		\begin{itemize}
			\item Apa risiko dan pelanggaran etis?
			\item Apa alternatif keputusannya?
			\item Jika Anda guru, apa tindakan Anda?
		\end{itemize}
	\end{columns}
\end{frame}

	
\begin{frame}[fragile]{Latihan Analisis Risiko dan Refleksi Etis (2)}
	\vspace{10pt}
	\begin{columns}[T]
		\column{0.48\textwidth}
		\textbf{Presentasi dan Debrief:}
		\begin{itemize}
			\item Bedakan bantuan AI dan kecurangan
			\item Tentukan batas pemakaian AI
			\item Bahas tanggung jawab guru
		\end{itemize}
		
		\vspace{3pt}
		\textbf{Refleksi Individu:}
		\begin{itemize}
			\item Apa kesadaran baru Anda?
			\item Perubahan praktik ke depan?
		\end{itemize}
		
		\column{0.48\textwidth}
		\textbf{Tugas Penutup:}
		\begin{itemize}
			\item Tulis pedoman/komitmen pribadi
		\end{itemize}
		
		\vspace{3pt}
		\textbf{Manfaat Latihan:}
		\begin{itemize}
			\item Bangun kesadaran etis dan kritis
			\item Kembangkan standar etika pribadi
			\item Guru sebagai teladan digital
		\end{itemize}
	\end{columns}
\end{frame}

	
	
		\end{document}
