\documentclass[11pt]{book}

\usepackage[utf8]{inputenc}
\usepackage{listingsutf8}
\lstset{inputencoding=utf8}

\usepackage{tabularx}
\usepackage[a4paper,outer=3cm,inner=3cm,top=3cm,bottom=3cm]{geometry}
\usepackage[colorlinks=true, linkcolor=black, citecolor=blue, urlcolor=blue]{hyperref}
\usepackage{graphicx}
\usepackage{xcolor}
\graphicspath{{images/}}
\usepackage{listings}
\usepackage{svg}

\usepackage{tikz}
\usetikzlibrary{shapes.geometric, arrows.meta, positioning}

\newcommand{\authors}[1]{{\chapterauthor{#1}\addtocontents{toc}{#1\par}}}

\renewcommand{\contentsname}{Daftar Isi}
\renewcommand{\chaptername}{Bab}
\renewcommand{\figurename}{Gambar}

\makeatletter
\newcommand{\chapterauthor}[1]{%
	{\parindent0pt\vspace*{-25pt}%
		\linespread{1.1}\large\scshape#1%
		\par\nobreak\vspace*{35pt}}
	\@afterheading%
}
\makeatother

\begin{document}
	
%	\begin{titlepage}
%		\centering
%		\vspace*{1cm}
%		
%		\Huge
%		\textbf{Pelatihan AI untuk Guru}
%		
%		\vspace{0.5cm}
%		
%		\LARGE
%		Universitas Pradita
%		
%		\vspace{1.5cm}
%		
%		\textit{Powered by ChatGPT}
%		
%		\vspace{2cm}
%		
%		\textbf{Alfa Yohannis}
%		
%		\vspace{0.8cm}
%		
%		\today
%		
%		\vfill
%	\end{titlepage}

\begin{titlepage}
	\begin{tikzpicture}[remember picture, overlay]
		% Gambar latar belakang atau logo
		\node[anchor=north west, inner sep=0pt] at (current page.north west){
			\includegraphics[width=4cm]{../figures/logo_universitas.png} % Ganti logo.png dengan path gambarmu
		};
		
		% Judul Tengah
		\node[anchor=center] at ([yshift=4cm]current page.center) {
			\begin{minipage}{0.8\textwidth}
				\centering
				{\Huge \textbf{Pelatihan AI untuk Guru}}\\[1cm]
				{\Large Penggunaan Large-language Model}\\[2cm]
				{\large \textbf{Nama Penulis}}\\[0.3cm]
				{\large NIM atau ID Penulis}\\[3cm]
				{\large Program Studi}\\[0.3cm]
				{\large Universitas atau Institusi}
			\end{minipage}
		};
		
		% Teks bawah
		\node[anchor=south] at (current page.south) {
			\begin{minipage}{\textwidth}
				\centering
				{\large \textbf{Bandung, April 2025}}
				\vspace{1cm}
			\end{minipage}
		};
	\end{tikzpicture}
\end{titlepage}
	
	% Contents Page
	\tableofcontents
	
	\chapter{Pengenalan LLM dalam Dunia Pendidikan}

\section{Apa itu Large Language Model (LLM)?}

Large Language Model (LLM) adalah jenis sistem kecerdasan buatan (AI) yang dirancang untuk memahami dan menghasilkan bahasa alami layaknya manusia. Model ini dilatih menggunakan kumpulan data teks dalam jumlah besar seperti buku, artikel, situs web, dan sumber lainnya untuk mempelajari pola, struktur, serta makna bahasa. Setelah melalui proses pelatihan, LLM mampu memberikan respons yang koheren dan relevan secara konteks terhadap masukan teks, sering kali menyerupai percakapan manusia.

LLM memproses masukan teks dengan mengubahnya menjadi \textbf{token}, yaitu potongan-potongan kecil dari teks seperti kata, suku kata, atau bahkan huruf tergantung konteksnya. Misalnya, kalimat "Saya makan nasi" akan dipecah menjadi token-token seperti "Saya", "makan", dan "nasi". Token inilah yang menjadi dasar bagi model untuk menganalisis dan memprediksi kelanjutan teks.

Token-token tersebut kemudian dianalisis menggunakan \textbf{jaringan saraf tiruan} (neural networks), yaitu sistem komputasi yang meniru cara kerja otak manusia dalam memproses informasi. Dalam konteks LLM, jaringan ini memiliki banyak lapisan yang bekerja untuk memahami hubungan antar token dan konteks di baliknya. Setiap lapisan bertugas mengenali pola-pola tertentu dalam bahasa, seperti tata bahasa, makna, dan gaya penulisan. Hasil dari proses ini adalah kemampuan model untuk memprediksi token berikutnya yang paling sesuai dalam sebuah kalimat atau paragraf.

Istilah "large" pada LLM merujuk pada jumlah parameter yang sangat besar (sering kali mencapai miliaran) yang dipelajari oleh model selama pelatihan. Parameter-parameter ini adalah angka-angka yang mengatur cara model membuat prediksi dan menyusun kalimat. Semakin banyak parameter, semakin besar pula kemampuan model dalam memahami konteks yang kompleks dan menangani berbagai topik serta tugas dengan baik.

Beberapa LLM yang dikenal luas di antaranya:
\begin{itemize}
	\item \textbf{ChatGPT} oleh OpenAI – umum digunakan untuk percakapan, pendamping belajar, dan bantuan penulisan.
	\item \textbf{Claude} oleh Anthropic – menekankan pada aspek keamanan dan keterjelasan interpretasi.
	\item \textbf{Gemini} oleh Google DeepMind – mengintegrasikan kemampuan LLM dalam produk-produk Google.
\end{itemize}

LLM termasuk dalam keluarga besar kecerdasan buatan generatif (generative AI), yang berfokus pada pembuatan konten, bukan hanya analisis. Dalam konteks pendidikan, hal ini membuka peluang hadirnya alat bantu inovatif yang dapat mendukung guru dalam menyiapkan materi, memberikan umpan balik, dan memenuhi kebutuhan belajar yang beragam.



\section{Bagaimana Cara Kerja LLM secara Sederhana}

Untuk memahami cara kerja Large Language Model (LLM) secara sederhana, perhatikan lima langkah utama yang ditampilkan pada Gambar~\ref{fig:diagram-llm}. Diagram tersebut memperlihatkan alur logis mulai dari data hingga keluaran berupa teks yang dihasilkan oleh model AI.

\tikzstyle{process} = [rectangle, rounded corners, minimum width=2cm, minimum height=1cm, text centered, align=center, draw=black, fill=blue!10,  font=\bfseries ]
\tikzstyle{arrow} = [thick,->,>=stealth]

\begin{figure}
	\centering
	\begin{tikzpicture}[node distance=.1cm, every node/.style={process}]
		\node (data) {Data\\Teks\\Besar};
		\node (training) [right of=data, xshift=2.6cm] {Pelatihan\\Model\\(Training)};
		\node (input) [right of=training, xshift=2.8cm] {Input dari\\Pengguna\\(Teks)};
		\node (token) [right of=input, xshift=3cm] {Tokenisasi\\dan Analisis\\Konteks};
		\node (output) [right of=token, xshift=3.2cm] {Prediksi dan\\Penyusunan\\Teks Output};
		
		\draw [arrow] (data) -- (training);
		\draw [arrow] (training) -- (input);
		\draw [arrow] (input) -- (token);
		\draw [arrow] (token) -- (output);
	\end{tikzpicture}
	\caption{Diagram Sederhana Cara Kerja LLM}
	\label{fig:diagram-llm}
\end{figure}

\textbf{1. Data Teks Besar.}  
Semua proses dimulai dari kumpulan data dalam skala besar yang terdiri dari miliaran kata—berasal dari buku, artikel ilmiah, berita, percakapan daring, dan berbagai sumber lain. Data ini digunakan sebagai dasar pembelajaran agar model mengenali struktur dan makna bahasa.

\textbf{2. Pelatihan Model (Training).}  
Model dilatih menggunakan data tersebut untuk memahami bagaimana satu kata mengikuti kata lain. Selama pelatihan, model belajar mengenali pola dalam kalimat dan menyesuaikan parameter internalnya agar mampu memprediksi token berikutnya secara akurat. Misalnya, jika diberi teks: "Air mengalir dari pegunungan ke...", maka model belajar menebak token berikutnya, seperti "sungai".

\textbf{3. Input dari Pengguna (Teks).}  
Setelah model selesai dilatih, ia dapat digunakan oleh pengguna melalui input teks. Input ini bisa berupa pertanyaan, perintah, atau topik tertentu. Misalnya, pengguna bisa menulis: "Tuliskan ringkasan tentang proses fotosintesis."

\textbf{4. Tokenisasi dan Analisis Konteks.}  
Input teks tersebut dipecah menjadi potongan-potongan kecil yang disebut token. Token ini kemudian dianalisis oleh jaringan saraf tiruan untuk memahami makna dan konteksnya. Model mempertimbangkan hubungan antar-token untuk memahami maksud keseluruhan dari teks yang dimasukkan.

\textbf{5. Prediksi dan Penyusunan Teks Output.}  
Berdasarkan pemahaman terhadap konteks, model mulai menyusun jawaban dengan memprediksi token-token berikutnya secara bertahap. Token-token ini disusun menjadi kalimat yang koheren, hingga terbentuklah teks yang utuh dan relevan dengan input yang diberikan.

Proses ini berlangsung dengan sangat cepat—dalam hitungan milidetik—dan memungkinkan pengguna menerima tanggapan seketika yang tampak alami dan masuk akal. Dengan mengenali tahapan-tahapan ini, penggunaan LLM dapat menjadi lebih terarah dan efektif dalam mendukung aktivitas pembelajaran.



\section{Penggunaan LLM di Kelas: Studi Kasus}

Large Language Model (LLM) seperti ChatGPT dapat menjadi asisten cerdas bagi guru dalam berbagai aktivitas pembelajaran. Dalam konteks kelas, LLM dapat digunakan untuk mempercepat persiapan materi, memperkaya pembelajaran, serta memberikan dukungan terhadap kebutuhan belajar yang beragam. Berikut ini adalah beberapa studi kasus penggunaan LLM secara praktis di lingkungan sekolah:

\textbf{1. Membuat Materi Ajar dan Soal.}  
Gunakan LLM untuk menghasilkan rencana pembelajaran, bahan ajar, dan latihan soal sesuai topik tertentu. Misalnya, untuk topik “Perubahan Iklim” dalam mata pelajaran Geografi, cukup berikan perintah seperti: \texttt{“Buat ringkasan materi dan latihan pilihan ganda untuk siswa SMA tentang perubahan iklim”}. Dalam hitungan detik, model akan menyusun konten tersebut lengkap dengan jawaban dan penjelasan.

\textbf{2. Menyederhanakan Konten Kompleks.}  
LLM mampu menyederhanakan bacaan ilmiah atau dokumen teknis menjadi versi yang lebih mudah dipahami. Ini sangat berguna untuk mendukung literasi sains atau saat membahas artikel berita ilmiah. Contohnya, artikel jurnal dapat disederhanakan dengan prompt seperti: \texttt{“Sederhanakan artikel ini untuk siswa kelas 10 dengan penjelasan yang mudah dimengerti”}.

\textbf{3. Memberikan Umpan Balik Otomatis.}  
Dalam tugas esai atau laporan, LLM dapat digunakan untuk memberi komentar konstruktif. Salin jawaban siswa lalu gunakan prompt seperti: \texttt{"Berikan umpan balik yang membangun atas tulisan ini ber\-da\-sar\-kan struktur, isi, dan tata bahasa"}. Ini sangat membantu menghemat waktu, ter\-u\-ta\-ma untuk kelas besar.

\textbf{4. Adaptasi Pembelajaran untuk Kebutuhan Khusus.}  
Gunakan LLM untuk membuat versi alternatif dari materi pembelajaran bagi siswa dengan kebutuhan khusus, seperti hambatan membaca atau siswa ESL (English as a Second Language). Misalnya, gunakan prompt: \texttt{“Buat versi teks ini dengan kalimat lebih pendek dan kosakata yang sederhana”}.

\textbf{5. Simulasi dan Diskusi Kelas.}  
Ciptakan skenario diskusi atau simulasi peran menggunakan LLM. Dalam pelajaran Sejarah, misalnya: \texttt{“Tuliskan percakapan i\-ma\-ji\-na\-tif antara Soekarno dan Mahatma Gandhi tentang kemerdekaan”}. Aktivitas ini dapat memicu diskusi kritis dan memperkaya kreativitas siswa.

Dengan berbagai cara tersebut, LLM menjadi alat bantu yang fleksibel dan adaptif untuk memperkaya pembelajaran. Meski begitu, penggunaannya tetap perlu pengawasan agar hasil yang digunakan tetap relevan, akurat, dan sesuai konteks.


\section{Demonstrasi Langsung}

Untuk memahami bagaimana Large Language Model (LLM) seperti ChatGPT bekerja dan dimanfaatkan dalam pembelajaran, ikuti demonstrasi langsung berikut. Amati bagaimana model merespons berbagai perintah (prompt), dan nilai secara kritis hasil yang diberikan.

\textbf{1. Lihat Alur Interaksi dengan LLM.} Jalankan web browser, lalu buka aplikasi web ChatGPT \url{https://chatgpt.com}. Perhatikan bagaimana antarmuka aplikasi digunakan untuk mengajukan pertanyaan, menyusun prompt yang efektif, dan menerima hasil dalam hitungan detik. Pahami bagaimana struktur prompt memengaruhi jenis jawaban yang dihasilkan.


\textbf{2. Cobalah Beberapa Prompt Berikut.} Berikut beberapa contoh prompt yang dapat dicoba secara langsung:
\begin{itemize}
	\item \texttt{“Buatkan rencana pelajaran selama 1 jam untuk topik fotosintesis ting\-kat SMA.”}
	\item \texttt{“Buat 5 soal pilihan ganda mengenai revolusi industri be\-ser\-ta ja\-wab\-an\-nya\-.”}
	\item \texttt{“Sederhanakan artikel berikut untuk siswa kelas 8.”}
	\item \texttt{“Tuliskan komentar umpan balik untuk esai siswa berikut.”}
\end{itemize}


\textbf{3. Evaluasi Hasil yang Dihasilkan.} Tinjau hasil dari masing-masing prompt. Apakah sesuai dengan konteks pembelajaran? Apakah ada bagian yang perlu diperbaiki atau disesuaikan? Gunakan penilaian profesional untuk memutuskan apakah konten tersebut siap digunakan atau perlu revisi.


\textbf{4. Eksplorasi Mandiri.} Gunakan waktu yang tersedia untuk mencoba membuat prompt sendiri sesuai mata pelajaran atau kebutuhan yang sedang dihadapi. Bandingkan hasil yang diperoleh dengan ekspektasi. Ubah gaya, tingkat kesulitan, atau format untuk melihat variasi respons.


\textbf{5. Simpan dan Dokumentasikan.} Jika menemukan hasil yang berguna, tangkap layar atau salin hasilnya sebagai dokumentasi. Gunakan sebagai referensi untuk eksperimen lanjutan atau bahan pelengkap kegiatan pembelajaran.


Demonstrasi ini dirancang untuk mendorong eksplorasi aktif dan reflektif terhadap potensi LLM, serta membiasakan penggunaan teknologi ini secara efektif, kreatif, dan etis.


\section*{Latihan Praktik: Mengeksplorasi Prompt ChatGPT}
\addcontentsline{toc}{section}{Latihan Praktik: Mengeksplorasi Prompt ChatGPT}
\begin{itemize}
	\item \textbf{Tujuan:} Memahami bagaimana struktur prompt memengaruhi respons AI.
	\item \textbf{Tugas:} Coba prompt berikut di ChatGPT atau LLM lainnya:  
	\begin{quote}
		\texttt{"Buat rencana pembelajaran 30 menit untuk pelajaran biologi SMA tentang fotosintesis."}
	\end{quote}
	\item \textbf{Tantangan:} Ubah prompt tersebut untuk:
	\begin{itemize}
		\item Menargetkan mata pelajaran lain (misalnya, matematika, sastra)
		\item Menyesuaikan tingkat kelas (misalnya, SMP vs. SMA)
		\item Meminta format alternatif (misalnya, kerja kelompok, flipped classroom)
		\item Silahkan bereksperimen dengan berbagai kondisi spesifik.
	\end{itemize}
\end{itemize}

	\chapter{Pembuatan Rencana Pembelajaran dan Lembar Kerja}

\section{Mengapa Perencanaan Pembelajaran yang Efisien Itu Penting}

Perencanaan pembelajaran merupakan fondasi utama dalam proses mengajar yang efektif. Rencana pembelajaran yang baik memungkinkan kegiatan belajar berlangsung terstruktur, terukur, dan sesuai dengan capaian kompetensi. Namun dalam praktiknya, banyak pendidik menghadapi keterbatasan waktu, sumber daya, dan keharusan untuk menyesuaikan materi dengan kebutuhan beragam peserta didik. Inilah alasan mengapa efisiensi dalam merancang pembelajaran menjadi sangat penting.

Dengan bantuan teknologi seperti Large Language Model (LLM), proses perencanaan dapat diotomatisasi sebagian. Hal ini berdampak langsung pada penghematan waktu dan energi dalam menyusun komponen pembelajaran, seperti tujuan pembelajaran, strategi penyampaian, serta soal evaluasi. Guru tidak perlu memulai dari nol setiap kali menyusun perangkat ajar, karena LLM dapat menghasilkan kerangka rencana pembelajaran hanya dengan memberikan deskripsi topik atau tujuan yang diinginkan.

Selain efisiensi waktu, pendekatan ini juga mendukung diferensiasi pembelajaran. Artinya, materi dan aktivitas belajar dapat dengan mudah dimodifikasi sesuai dengan tingkat kemampuan siswa, gaya belajar, atau kebutuhan khusus. LLM mampu menyesuaikan versi materi untuk siswa pemula, menengah, hingga lanjutan—bahkan menyusun variasi kegiatan yang sesuai untuk kelas yang heterogen.

Oleh karena itu, penggunaan alat bantu berbasis AI seperti LLM tidak hanya menjawab tantangan keterbatasan waktu, tetapi juga membantu guru dalam merancang pembelajaran yang lebih inklusif, adaptif, dan responsif terhadap keragaman peserta didik. Pendekatan ini memungkinkan lebih banyak waktu dialokasikan untuk interaksi bermakna di kelas dan pembimbingan individu, yang merupakan inti dari pendidikan yang berkualitas.


\section{Menyusun Rencana Pembelajaran Menggunakan LLM}

Large Language Model (LLM) seperti ChatGPT dapat membantu menyusun rencana pembelajaran secara cepat dan fleksibel. Dengan hanya memberikan deskripsi topik, jenjang pendidikan, serta tujuan pembelajaran, model dapat menghasilkan struktur rencana yang mencakup komponen utama seperti tujuan, langkah kegiatan, media, dan metode penilaian. Hal ini sangat berguna bagi pendidik yang membutuhkan inspirasi awal, kerangka kerja, atau variasi pendekatan dalam menyampaikan materi.

Sebagai contoh, berikut prompt yang dapat digunakan:
\begin{quote}
	\centering
		\texttt{“Buatkan rencana pembelajaran berdurasi 1 jam untuk siswa SMA kelas X tentang sistem pernapasan manusia, lengkap dengan tujuan pembelajaran, kegiatan inti, dan evaluasi.”}
\end{quote}

Hasil dari prompt tersebut biasanya mencakup:
\begin{itemize}
	\item \textbf{Tujuan Pembelajaran:} Siswa mampu menjelaskan proses pernapasan dan mengidentifikasi organ yang terlibat.
	\item \textbf{Pendahuluan:} Guru menanyakan pertanyaan pemantik dan memutar video pendek.
	\item \textbf{Kegiatan Inti:} Diskusi kelompok berdasarkan teks bacaan dan presentasi hasil diskusi.
	\item \textbf{Penutup:} Refleksi singkat dan kuis 5 soal pilihan ganda.
\end{itemize}

Rencana ini dapat diedit langsung oleh pendidik sesuai konteks kelas, termasuk menambahkan kegiatan berbasis proyek, menyisipkan asesmen formatif, atau menyusun alternatif aktivitas untuk siswa dengan kebutuhan khusus. Misalnya, jika terdapat siswa dengan hambatan penglihatan, bagian visualisasi dapat diganti dengan narasi atau audio.

LLM juga dapat dimanfaatkan untuk membuat versi rencana pembelajaran dalam bahasa Inggris, atau untuk kelas bilingual. Dengan menyisipkan konteks tambahan seperti gaya mengajar, durasi pembelajaran, atau pendekatan pedagogis tertentu (misalnya pembelajaran berbasis masalah atau pembelajaran tematik), hasil rencana yang dihasilkan bisa lebih personal dan kontekstual.

Penting untuk diingat bahwa hasil dari LLM bersifat usulan awal. Tinjauan dan penyesuaian tetap perlu dilakukan agar rencana pembelajaran sesuai dengan kurikulum yang berlaku, kebutuhan peserta didik, serta karakteristik kelas. Dengan pendekatan kolaboratif antara manusia dan AI, proses perencanaan menjadi lebih cepat tanpa mengorbankan kualitas.

\section{Membuat Soal dan Kuis secara Otomatis}

Pembuatan soal dan kuis merupakan salah satu kegiatan rutin yang memerlukan waktu dan ketelitian. Guru perlu memastikan bahwa soal yang disusun relevan dengan tujuan pembelajaran, memiliki tingkat kesulitan yang sesuai, serta mampu mengukur pemahaman peserta didik secara adil dan terstruktur. Dengan bantuan Large Language Model (LLM), proses ini dapat dilakukan secara otomatis dan efisien, tanpa mengorbankan variasi dan kualitas soal.

LLM seperti ChatGPT dapat digunakan untuk menghasilkan berbagai jenis soal, antara lain:
\begin{itemize}
	\item \textbf{Pilihan Ganda} – Cocok untuk asesmen cepat dan diagnosis awal pemahaman konsep.
	\item \textbf{Isian Singkat} – Mengukur kemampuan mengingat dan menerapkan informasi faktual.
	\item \textbf{Uraian Pendek} – Mendorong siswa menjelaskan proses, membandingkan konsep, atau memberikan pendapat.
	\item \textbf{Soal Benar/Salah dan Menjodohkan} – Berguna untuk latihan ringan dan kuis interaktif.
\end{itemize}

Contoh prompt yang dapat digunakan:
\begin{quote}
	\centering
	\texttt{“Buatkan 5 soal pilihan ganda tentang revolusi industri untuk siswa kelas XI IPS lengkap dengan opsi jawaban dan penjelasan jawaban yang benar.”}
\end{quote}

Hasil yang diberikan oleh LLM umumnya sudah cukup siap digunakan, lengkap dengan:
\begin{itemize}
	\item Nomor soal dan teks pertanyaan
	\item Empat pilihan jawaban (A–D)
	\item Indikasi jawaban yang benar
	\item Penjelasan singkat mengapa jawaban tersebut benar
\end{itemize}

Soal-soal ini dapat langsung dicetak atau dimasukkan ke dalam platform pembelajaran daring (seperti Google Forms, Moodle, atau Kahoot). Jika diperlukan, prompt dapat diubah agar LLM menyesuaikan gaya bahasa, tingkat kesulitan, atau tipe soal tertentu (misalnya “buat soal HOTS” atau “buat soal dengan konteks kehidupan sehari-hari”).

Selain untuk evaluasi sumatif, LLM juga bisa digunakan untuk membuat soal latihan harian, tugas remedial, maupun latihan mandiri yang berbeda-beda untuk setiap kelompok siswa. Hal ini memungkinkan pendekatan diferensiasi secara lebih praktis.

Namun, penting untuk tetap memeriksa dan menyunting hasil soal yang dihasilkan. Meskipun akurat secara umum, kadang ditemukan pertanyaan ambigu, opsi jawaban yang tidak setara, atau kekeliruan konsep kecil yang perlu diperbaiki. Dengan menyandingkan kecepatan LLM dan kepekaan pedagogis guru, pembuatan soal menjadi lebih cepat, bervariasi, dan tetap relevan secara pedagogis.


\section{Menyesuaikan Lembar Kerja untuk Beragam Kebutuhan Belajar}

Setiap kelas terdiri dari peserta didik dengan latar belakang, kemampuan, dan gaya belajar yang beragam. Oleh karena itu, lembar kerja yang digunakan dalam kegiatan belajar perlu disesuaikan agar semua siswa dapat mengakses materi secara optimal. Penyesuaian ini mencakup tidak hanya tingkat kesulitan soal, tetapi juga format penyajian, bahasa yang digunakan, dan konteks yang relevan bagi peserta didik.

Dengan bantuan Large Language Model (LLM), penyesuaian lembar kerja dapat dilakukan dengan cepat dan fleksibel. Guru cukup memberikan instruksi tambahan pada prompt untuk menyesuaikan konten, misalnya: “buat versi sederhana”, “gunakan kalimat pendek”, atau “ubah ke gaya narasi cerita untuk siswa SD”. Hasil yang dihasilkan oleh model dapat menjadi titik awal dalam merancang lembar kerja yang inklusif.

Beberapa bentuk penyesuaian yang dapat dilakukan antara lain:

\begin{itemize}
	\item \textbf{Tingkat Kesulitan:} Lembar kerja untuk siswa tingkat dasar dapat dibuat dengan pertanyaan langsung dan contoh konkret, sementara untuk siswa tingkat lanjut dapat mencakup analisis, perbandingan, atau eksplorasi ide terbuka.
	\item \textbf{Bahasa dan Format:} Gunakan kalimat sederhana, paragraf pendek, dan visual pendukung untuk siswa dengan kebutuhan khusus atau keterbatasan literasi. LLM dapat membantu menyusun versi teks yang lebih mudah dibaca tanpa kehilangan makna inti.
	\item \textbf{Kebutuhan Khusus:} Untuk siswa dengan hambatan belajar, seperti disleksia atau kesulitan fokus, LLM dapat menyusun teks dalam format bullet-point, memperbanyak ruang kosong antarbaris, atau menyertakan petunjuk dalam bentuk simbol dan warna.
	\item \textbf{Gaya Belajar:} Siswa dengan gaya belajar visual dapat dibantu dengan lembar kerja berbasis gambar, diagram, atau ilustrasi. Untuk gaya auditori, LLM dapat membantu membuat transkrip audio atau narasi yang mendampingi isi lembar kerja.
\end{itemize}

Contoh prompt yang dapat digunakan:
\begin{quote}
	\centering
	\texttt{“Buatkan lembar kerja tentang daur air untuk siswa kelas 4 SD, dengan teks pendek, gambar pendukung, dan soal pilihan ganda yang mudah dimengerti.”}
\end{quote}

Selain itu, LLM juga dapat membantu membuat beberapa versi lembar kerja untuk topik yang sama—misalnya satu untuk kelompok cepat, satu untuk kelompok sedang, dan satu untuk kelompok dengan kebutuhan pembelajaran tambahan. Ini memungkinkan guru menerapkan strategi pembelajaran terdiferensiasi dengan lebih mudah dan konsisten.

Dengan demikian, penggunaan LLM tidak hanya membantu mempercepat pembuatan lembar kerja, tetapi juga memastikan bahwa materi yang diberikan dapat diakses oleh semua siswa, tanpa memandang perbedaan kemampuan atau gaya belajarnya. Pendekatan ini memperkuat prinsip inklusivitas dalam pembelajaran.


\section{Latihan Membuat Lembar Kerja dan Kuis}

Setelah mempelajari berbagai potensi penggunaan LLM untuk perencanaan pembelajaran, bagian ini memberikan kesempatan untuk praktik langsung. Tujuannya adalah agar setiap pendidik tidak hanya memahami konsep, tetapi juga mampu menerapkannya secara mandiri dalam kegiatan belajar-mengajar.

Latihan ini berfokus pada pembuatan lembar kerja dan kuis dengan menggunakan LLM seperti ChatGPT. Setiap bagian latihan dirancang untuk menuntun secara bertahap, mulai dari perencanaan topik hingga evaluasi hasil.

\textbf{1. Tentukan Topik dan Tujuan Pembelajaran.}  
Pilih salah satu topik dari mata pelajaran yang diajarkan. Tetapkan tujuan pembelajaran secara jelas dan spesifik. Contoh:
\begin{quote}
	\centering
	\texttt{“Siswa dapat menjelaskan proses daur air dan mengidentifikasi tahapan-tahapannya.”}
\end{quote}

\textbf{2. Gunakan Prompt untuk Membuat Lembar Kerja.}  
Masukkan perintah (prompt) ke dalam LLM, seperti:
\begin{quote}
	\centering
	\texttt{“Buatkan lembar kerja tentang daur air untuk siswa kelas 5 SD, gunakan kalimat sederhana, sertakan 3 soal pilihan ganda dan 1 soal uraian pendek.”}
\end{quote}
Bandingkan hasilnya dengan kebutuhan kelas. Ubah jika perlu, lalu buat dua versi: satu untuk siswa dengan pemahaman cepat, dan satu lagi untuk siswa yang membutuhkan pendekatan lebih sederhana.

\textbf{3. Buat Kuis atau Latihan Evaluasi.}  
Gunakan LLM untuk menghasilkan soal evaluasi. Misalnya:
\begin{quote}
	\centering
	\texttt{“Buat 5 soal pilihan ganda dan 3 soal benar-salah tentang sifat-sifat cahaya untuk siswa kelas 6.”}
\end{quote}
Periksa apakah hasil kuis mencakup kunci jawaban dan penjelasan. Ubah redaksi bila ada kalimat yang kurang sesuai dengan gaya bahasa di kelas.

\textbf{4. Evaluasi Hasil dan Kelayakan Penggunaan.}  
Tinjau kembali hasil yang dibuat: apakah sesuai dengan kebutuhan peserta didik? Apakah tingkat kesulitan dan bahasa yang digunakan tepat? Apa yang perlu disesuaikan sebelum digunakan dalam proses pembelajaran?

\textbf{5. Simpan dan Bagikan dengan Rekan.}  
Jika hasilnya dirasa layak, simpan sebagai arsip materi ajar. Jika memungkinkan, bagikan kepada rekan pendidik untuk dijadikan referensi bersama atau kolaborasi materi.

Latihan ini bertujuan untuk mengasah keterampilan dalam mendesain materi pembelajaran yang responsif, efisien, dan relevan dengan perkembangan teknologi pengajaran berbasis AI.


\section*{Latihan Praktik: Mendesain Materi dan Kuis Adaptif}
\addcontentsline{toc}{section}{Latihan Praktik: Mendesain Materi dan Kuis Adaptif}
\begin{itemize}
	\item \textbf{Tujuan:} Membuat lembar kerja dan kuis yang sesuai dengan topik serta kebutuhan peserta didik.
	\item \textbf{Tugas:} Gunakan LLM untuk menghasilkan:
	\begin{itemize}
		\item Rencana pembelajaran berdurasi 1 jam untuk topik pilihan.
		\item Lima soal pilihan ganda beserta jawabannya.
		\item Lembar kerja yang disesuaikan untuk siswa dengan kebutuhan belajar khusus (misalnya: ringan teks, berpola gambar, atau bertingkat kesulitan).
	\end{itemize}
\end{itemize}

	\include{chapter03}
	\include{chapter04}
	\chapter{AI untuk Tugas Administratif dan Komunikasi}

\section{Mengapa Dukungan AI Penting dalam Tugas Administratif}
% Penjelasan beban administratif guru dan bagaimana AI dapat menghemat waktu serta meningkatkan efisiensi

\section{Menyusun Email dan Surat untuk Orang Tua}
% Contoh penggunaan LLM untuk menyusun email pemberitahuan, surat undangan, atau catatan perkembangan siswa

\section{Membuat Laporan Akademik dan Nonakademik secara Otomatis}
% Cara menyusun ringkasan hasil belajar, laporan kegiatan, atau refleksi siswa menggunakan LLM

\section{Menyusun Surat Resmi dan Dokumen Sekolah}
% Template prompt untuk membuat surat tugas, undangan resmi, atau notulensi rapat

\section{Merangkum Catatan dan Observasi Harian}
% Cara menggunakan LLM untuk merangkum catatan guru, hasil diskusi, atau poin-poin penting dari pertemuan

\section{Latihan Menyusun Komunikasi dan Dokumen Sekolah}
% Aktivitas praktik membuat surat, email, dan laporan dari masukan teks sederhana

\section*{Latihan Praktik: Dokumen dan Komunikasi Otomatis}
\addcontentsline{toc}{section}{Latihan Praktik: Dokumen dan Komunikasi Otomatis}
\begin{itemize}
	\item \textbf{Tujuan:} Membuat dokumen komunikasi dan laporan dengan cepat dan tepat menggunakan LLM.
	\item \textbf{Tugas:}
	\begin{itemize}
		\item Buat surat pemberitahuan kegiatan kelas kepada orang tua.
		\item Susun laporan perkembangan belajar berdasarkan poin-poin hasil observasi siswa.
		\item Gunakan LLM untuk menuliskan notulensi rapat dari daftar topik yang telah dibahas.
	\end{itemize}
\end{itemize}

	\chapter{Etika dalam Penggunaan AI: Apa yang Bisa Salah?}

\section{Mengapa Etika Penting dalam Penggunaan AI di Pendidikan}
% Penjelasan tentang potensi risiko AI dan pentingnya sikap kritis terhadap hasil yang dihasilkan oleh LLM

\section{Bias dalam AI dan Dampaknya pada Pembelajaran}
% Bagaimana bias dalam data pelatihan dapat menyebabkan hasil yang tidak netral atau diskriminatif

\section{Risiko Misinformasi dan Fakta yang Salah}
% Penjelasan bahwa LLM bisa menghasilkan informasi yang tampak meyakinkan tetapi tidak akurat

\section{Ketergantungan Berlebih terhadap AI}
% Risiko menurunnya pemikiran kritis dan inisiatif siswa maupun guru jika terlalu mengandalkan AI

\section{Plagiarisme dan Keaslian Karya}
% Tantangan dalam membedakan karya orisinal dengan hasil AI serta bagaimana mengedukasi siswa tentang integritas akademik

\section{Privasi dan Keamanan Data}
% Bahaya menyisipkan data pribadi atau sensitif ke dalam sistem AI yang terhubung dengan internet

\section{Menggunakan AI secara Aman dan Bertanggung Jawab di Kelas}
% Praktik-praktik terbaik untuk memandu penggunaan AI secara bijak, aman, dan etis oleh guru dan siswa

\section{Latihan: Analisis Risiko dan Refleksi Etis}
% Aktivitas praktik berupa studi kasus, diskusi, dan pengambilan keputusan etis

\section*{Latihan Praktik: Etika dan Pengambilan Keputusan dalam Penggunaan AI}
\addcontentsline{toc}{section}{Latihan Praktik: Etika dan Pengambilan Keputusan dalam Penggunaan AI}
\begin{itemize}
	\item \textbf{Tujuan:} Mengenali potensi risiko penggunaan AI dan membangun sikap reflektif dalam pengambilan keputusan.
	\item \textbf{Tugas:}
	\begin{itemize}
		\item Analisis kasus: “Siswa mengumpulkan tugas hasil ChatGPT tanpa revisi” – apa langkah bijak yang dapat diambil?
		\item Buat daftar panduan penggunaan AI yang aman dan etis untuk siswa di kelas.
		\item Diskusikan batasan penggunaan AI untuk tugas, ujian, dan interaksi guru-siswa.
	\end{itemize}
\end{itemize}


%	09:00 – 10:00	1. Introduction to LLMs in Education	Overview of AI & LLMs like ChatGPT, examples in teaching, demo session
%	10:00 – 11:00	2. Smart Lesson & Worksheet Creation	Generate lessons, quizzes, and custom worksheets for diverse learners
%	11:00 – 12:00	☕ Break	Refresh & informal discussion
%	12:00 – 13:00	3. Personalised Feedback & Assessment	Use AI to grade, give feedback, and build rubrics or self-assessment tools
%	13:00 – 14:00	4. AI-Powered Research & Content Summarisation	Use LLMs to summarise articles, find teaching resources, or generate reports
%	14:00 – 15:00	🍽️ Lunch Break	Relax and network
%	15:00 – 16:00	5. AI for Admin Tasks & Communication	Drafting parent emails, reports, letters, and summarising notes
%	16:00 – 17:00	6. Ethics in AI: What Can Go Wrong?	Bias, misinformation, overreliance, plagiarism, privacy, and safe classroom use
	\backmatter
	\addcontentsline{toc}{chapter}{Daftar Pustaka}
	\bibliography{references}
	
\end{document}