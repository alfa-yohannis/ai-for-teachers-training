\chapter{AI untuk Tugas Administratif dan Komunikasi}

\section{Mengapa Dukungan AI Penting dalam Tugas Administratif}

Selain mengajar dan mendampingi siswa, guru juga memikul tanggung jawab administratif yang tidak sedikit. Kegiatan seperti menyusun laporan siswa, membuat surat undangan orang tua, merangkum notulensi rapat, mengisi dokumen perencanaan, hingga menyusun refleksi pembelajaran, memakan waktu yang cukup besar di luar jam tatap muka. Beban administratif ini sering kali mengurangi waktu yang seharusnya dapat digunakan untuk merancang kegiatan belajar yang lebih kreatif dan mempersonalisasi pembelajaran.

Dalam konteks inilah teknologi kecerdasan buatan, khususnya Large Language Models (LLM) seperti ChatGPT, dapat menjadi mitra strategis bagi guru dalam menyederhanakan dan mempercepat tugas-tugas administratif yang bersifat rutin dan berbasis teks. LLM mampu menghasilkan draft dokumen secara otomatis, menyusun email yang sopan dan efektif, merangkum hasil rapat, serta menyesuaikan gaya bahasa sesuai kebutuhan komunikasi formal sekolah.

\textbf{Jenis tugas administratif yang dapat dibantu oleh AI antara lain:}
\begin{itemize}
	\item Menulis email kepada orang tua atau kolega
	\item Menyusun laporan perkembangan belajar siswa
	\item Membuat undangan kegiatan sekolah
	\item Menulis surat keterangan atau rekomendasi
	\item Merangkum poin-poin penting dari catatan rapat atau hasil diskusi
\end{itemize}

\textbf{Contoh prompt untuk mendukung tugas administratif:}

\begin{quote}
	\centering
	\texttt{"Tulis email resmi kepada orang tua untuk mengundang mereka menghadiri pertemuan wali kelas minggu depan, gunakan bahasa sopan dan ringkas."}
\end{quote}

\begin{quote}
	\centering
	\texttt{"Susun laporan perkembangan belajar siswa bernama Dina berdasarkan catatan berikut: aktif saat diskusi, cepat memahami materi, perlu peningkatan dalam ketelitian."}
\end{quote}

Dari prompt seperti ini, LLM dapat langsung menghasilkan teks yang siap digunakan atau hanya memerlukan sedikit penyesuaian. Hal ini sangat membantu, terutama saat guru harus menangani puluhan siswa dalam waktu yang terbatas.

\textbf{Manfaat utama dukungan AI dalam konteks administratif:}
\begin{itemize}
	\item \textbf{Efisiensi waktu:} Tugas yang biasanya memakan waktu 30–60 menit bisa diselesaikan dalam hitungan menit.
	\item \textbf{Konsistensi gaya komunikasi:} LLM dapat menjaga kesopanan, struktur, dan keseragaman isi dalam dokumen resmi.
	\item \textbf{Fleksibilitas tinggi:} Guru dapat menyesuaikan format, nada, dan isi hanya dengan mengubah instruksi dalam prompt.
	\item \textbf{Reduksi beban kerja mental:} Guru dapat lebih fokus pada aspek pedagogis dan hubungan dengan siswa dibanding tenggelam dalam pekerjaan administratif.
\end{itemize}

Penting untuk dicatat bahwa hasil dari AI tetap memerlukan verifikasi dan penyesuaian oleh manusia. LLM dapat menghasilkan draft yang sangat membantu, tetapi keputusan akhir dan sentuhan personal tetap berada di tangan guru.

Dengan menggunakan AI secara bijak, guru dapat mengembalikan lebih banyak waktu dan energi untuk hal yang paling penting dalam pendidikan: membangun hubungan bermakna dan pengalaman belajar yang mendalam bagi siswa.


\section{Menyusun Email dan Surat untuk Orang Tua}

Komunikasi yang efektif antara sekolah dan orang tua merupakan komponen penting dalam membangun ekosistem pembelajaran yang positif dan kolaboratif. Guru sering kali perlu menyampaikan berbagai informasi melalui email atau surat, seperti undangan pertemuan, pemberitahuan kegiatan, perkembangan belajar siswa, atau klarifikasi terhadap isu tertentu. Tantangannya, membuat komunikasi tertulis yang sopan, jelas, dan efisien memerlukan waktu dan perhatian yang tidak sedikit—terutama ketika jumlah siswa yang ditangani cukup banyak.

Dengan bantuan Large Language Model (LLM) seperti ChatGPT, guru dapat menyusun draft email dan surat secara cepat dan konsisten. LLM mampu memahami konteks komunikasi yang diinginkan dan menghasilkan teks dengan struktur yang rapi, bahasa yang formal namun tetap hangat, serta gaya komunikasi yang sesuai dengan norma pendidikan.

\textbf{Jenis komunikasi yang dapat disusun dengan bantuan LLM:}
\begin{itemize}
	\item \textbf{Email pemberitahuan kegiatan} (misalnya: lomba sekolah, kunjungan belajar, kegiatan pramuka)
	\item \textbf{Surat undangan pertemuan orang tua siswa}
	\item \textbf{Catatan perkembangan siswa secara individual}
	\item \textbf{Tanggapan terhadap pertanyaan atau kekhawatiran orang tua}
	\item \textbf{Surat keterangan atau klarifikasi administrasi sekolah}
\end{itemize}

\textbf{Contoh prompt untuk membuat email atau surat:}

\begin{quote}
	\centering
	\texttt{"Tulis email resmi kepada orang tua siswa kelas 7 untuk mengundang mereka menghadiri pertemuan wali kelas yang akan dilaksanakan pada hari Jumat, pukul 15.00 di ruang aula. Sertakan salam pembuka dan penutup yang sopan."}
\end{quote}

\textbf{Contoh hasil dari LLM:}

\begin{quote}
	Yth. Bapak/Ibu Orang Tua/Wali Siswa Kelas 7,\\
	
	Dengan hormat,\\
	
	Kami mengundang Bapak/Ibu untuk menghadiri pertemuan wali kelas yang akan dilaksanakan pada:
	
	\begin{itemize}
		\item Hari/Tanggal: Jumat, 12 Mei 2025
		\item Waktu: Pukul 15.00 WIB
		\item Tempat: Aula Sekolah
	\end{itemize}
	
	Pertemuan ini bertujuan untuk membahas perkembangan belajar siswa serta menjalin komunikasi yang lebih erat antara orang tua dan wali kelas. Kehadiran Bapak/Ibu sangat kami harapkan.
	
	Atas perhatian dan kerja sama Bapak/Ibu, kami ucapkan terima kasih.
	
	Hormat kami,\\
	Wali Kelas 7
\end{quote}

\textbf{Keunggulan penggunaan LLM dalam menyusun komunikasi dengan orang tua:}
\begin{itemize}
	\item \textbf{Efisiensi}: Menghemat waktu dalam membuat surat atau email yang seragam untuk banyak siswa.
	\item \textbf{Konsistensi}: Format dan gaya bahasa yang seragam membantu menjaga profesionalisme sekolah.
	\item \textbf{Kustomisasi cepat}: Guru dapat dengan mudah menyesuaikan isi surat berdasarkan kebutuhan atau situasi khusus masing-masing siswa.
\end{itemize}

Meskipun hasil LLM sudah sangat membantu, guru tetap dianjurkan untuk meninjau dan menyunting hasil akhir agar sesuai dengan konteks lokal, kebijakan sekolah, dan kebutuhan komunikasi yang lebih personal. Dengan begitu, penggunaan AI tidak menggantikan peran guru, tetapi mendukungnya secara signifikan dalam membangun hubungan yang harmonis antara sekolah dan rumah.


\section{Membuat Laporan Akademik dan Nonakademik secara Otomatis}

Laporan merupakan bagian penting dari dokumentasi pembelajaran yang berfungsi sebagai refleksi kemajuan siswa, evaluasi kegiatan sekolah, serta bahan komunikasi antara guru, siswa, dan orang tua. Di sisi lain, penyusunan laporan—baik akademik maupun nonakademik—merupakan pekerjaan yang memerlukan waktu, ketelitian, dan konsistensi. Terutama bagi guru yang menangani banyak siswa atau kegiatan sekaligus, membuat laporan bisa menjadi beban tambahan yang signifikan.

Dengan bantuan Large Language Model (LLM) seperti ChatGPT, proses pembuatan laporan dapat dilakukan secara otomatis dan efisien. Cukup dengan memberikan poin-poin utama, catatan observasi, atau format dasar, LLM dapat menghasilkan narasi lengkap yang siap digunakan sebagai laporan akademik, laporan kegiatan, maupun refleksi pembelajaran.

\textbf{Jenis laporan yang dapat dibantu oleh LLM:}
\begin{itemize}
	\item \textbf{Laporan hasil belajar siswa} (berdasarkan nilai, observasi, atau jurnal guru)
	\item \textbf{Laporan kegiatan kelas/sekolah} (misalnya: kunjungan industri, kerja kelompok, lomba)
	\item \textbf{Refleksi siswa atau refleksi kelas} (untuk dokumentasi Kurikulum Merdeka)
	\item \textbf{Laporan perkembangan karakter atau sikap}
\end{itemize}

\textbf{Contoh prompt untuk menyusun laporan akademik:}

\begin{quote}
	\centering
	\texttt{"Susun laporan hasil belajar siswa bernama Andi berdasarkan catatan berikut: memahami materi dengan baik, aktif bertanya, hasil ulangan stabil, perlu ditingkatkan dalam kerja kelompok."}
\end{quote}

\textbf{Contoh hasil dari LLM:}

\begin{quote}
	Andi menunjukkan pemahaman yang baik terhadap materi pelajaran. Ia aktif bertanya dan terlibat dalam diskusi kelas. Hasil ulangan harian tergolong stabil dan menunjukkan konsistensi. Namun demikian, kemampuan kerja sama dalam kelompok masih dapat ditingkatkan, terutama dalam hal pembagian tugas dan komunikasi antarteman. Secara umum, Andi menunjukkan kemajuan yang positif.
\end{quote}

\textbf{Contoh prompt untuk menyusun laporan kegiatan:}

\begin{quote}
	\centering
	\texttt{"Buatkan laporan kegiatan kunjungan ke museum untuk siswa kelas 6 berdasarkan catatan berikut: antusias saat tur, mengikuti arahan, mengerjakan lembar kerja, diskusi reflektif berjalan lancar."}
\end{quote}

\textbf{Contoh hasil laporan:}

\begin{quote}
	Kunjungan ke museum pada hari Kamis berjalan dengan lancar dan menyenangkan. Siswa menunjukkan antusiasme yang tinggi selama mengikuti tur edukatif, mendengarkan penjelasan pemandu, dan aktif mengajukan pertanyaan. Seluruh siswa menyelesaikan lembar kerja dengan baik dan terlibat dalam diskusi reflektif di akhir kegiatan. Kegiatan ini tidak hanya memperluas pengetahuan siswa, tetapi juga melatih kedisiplinan dan rasa ingin tahu mereka.
\end{quote}

\textbf{Manfaat menggunakan LLM untuk membuat laporan otomatis:}
\begin{itemize}
	\item \textbf{Menghemat waktu secara signifikan}, terutama untuk laporan rutin dan repetitif
	\item \textbf{Menjamin konsistensi bahasa dan struktur}, terutama dalam laporan resmi
	\item \textbf{Memberikan alternatif redaksi dan variasi gaya penulisan}, sesuai kebutuhan
	\item \textbf{Meningkatkan kualitas refleksi}, baik dari guru maupun siswa, dengan bantuan narasi yang lebih runtut dan mendalam
\end{itemize}

Meski begitu, hasil laporan dari LLM tetap perlu dikaji ulang oleh guru untuk menjamin akurasi, nuansa kontekstual, dan kesesuaian dengan kebijakan sekolah. Penggunaan AI bukan untuk menggantikan peran reflektif guru, melainkan sebagai alat bantu produktivitas yang mempercepat dokumentasi dan membuka ruang bagi lebih banyak interaksi bermakna dalam proses belajar.

\section{Menyusun Surat Resmi dan Dokumen Sekolah}

Dalam lingkungan pendidikan, surat resmi dan dokumen administratif merupakan instrumen penting yang mendukung kelancaran komunikasi, koordinasi, dan pencatatan kegiatan. Surat tugas, undangan resmi, notulensi rapat, hingga pengumuman kegiatan sekolah memerlukan penulisan yang formal, terstruktur, dan konsisten. Namun, banyak guru dan staf sekolah yang kesulitan menyusun dokumen-dokumen ini karena keterbatasan waktu, pengalaman administratif, atau beban kerja yang padat.

Dengan bantuan Large Language Model (LLM) seperti ChatGPT, proses penyusunan surat dan dokumen sekolah dapat dilakukan dengan lebih efisien. LLM mampu menghasilkan draf surat resmi dengan format yang tepat, bahasa yang sopan, dan informasi yang lengkap, hanya dengan memberikan instruksi (prompt) singkat mengenai isi dan tujuan surat.

\textbf{Jenis surat dan dokumen yang dapat disusun menggunakan LLM:}
\begin{itemize}
	\item Surat tugas guru atau panitia
	\item Surat undangan rapat atau kegiatan sekolah
	\item Notulensi rapat internal guru, wali kelas, atau komite sekolah
	\item Surat pengumuman kepada siswa atau orang tua
	\item Surat permohonan izin atau kerjasama
\end{itemize}

\textbf{Contoh prompt untuk surat tugas:}

\begin{quote}
	\centering
	\texttt{"Buatkan surat tugas resmi untuk guru bernama Ibu Ratna yang akan menjadi pendamping lomba OSN tingkat kota pada tanggal 22-24 Mei 2025 di SMAN 5. Gunakan bahasa formal dan sertakan kop surat."}
\end{quote}

\textbf{Contoh hasil dari LLM:}

\begin{quote}
	\textbf{DINAS PENDIDIKAN KOTA XYZ}\\
	\textbf{SMP NEGERI 3 XYZ}\\
	Jl. Merdeka No. 10, XYZ 12345\\
	Telp. (021) 4567890 \\[1em]
	
	\textbf{SURAT TUGAS}\\
	Nomor: 421.2/STM/05/2025 \\[1em]
	
	Kepala SMP Negeri 3 XYZ menugaskan:
	
	\textbf{Nama:} Ibu Ratna Yuliani, S.Pd \\
	\textbf{Jabatan:} Guru IPA
	
	Untuk mendampingi siswa dalam kegiatan Lomba Olimpiade Sains Nasional (OSN) tingkat kota yang dilaksanakan pada:
	
	\begin{itemize}
		\item Tanggal: 22–24 Mei 2025
		\item Tempat: SMAN 5 Kota XYZ
	\end{itemize}
	
	Demikian surat tugas ini dibuat untuk dilaksanakan sebagaimana mestinya.
	
	XYZ, 15 Mei 2025 \\
	Kepala Sekolah,\\
	\hspace*{1em} \\
	\textbf{Drs. Hadi Santoso, M.Pd}
\end{quote}

\textbf{Contoh prompt untuk membuat notulensi rapat:}

\begin{quote}
	\centering
	\texttt{"Buatkan notulensi rapat guru berdasarkan poin-poin berikut: evaluasi hasil ujian tengah semester, pembagian tugas piket ujian akhir, dan persiapan kegiatan class meeting."}
\end{quote}

\textbf{Manfaat menggunakan LLM untuk menyusun surat dan dokumen resmi:}
\begin{itemize}
	\item Menghemat waktu dan mengurangi beban administratif guru dan staf
	\item Memastikan penggunaan format dan bahasa resmi yang sesuai standar
	\item Memudahkan pembuatan surat massal dengan isi yang dapat disesuaikan
	\item Meningkatkan dokumentasi dan pencatatan kegiatan sekolah secara rapi dan profesional
\end{itemize}

Meskipun hasil dari LLM dapat langsung digunakan, disarankan tetap melakukan penyuntingan akhir untuk menyesuaikan dengan kebijakan institusi, kebutuhan kontekstual, dan kelengkapan informasi spesifik. Dengan pendekatan ini, LLM menjadi alat bantu administratif yang sangat bermanfaat di lingkungan sekolah, tanpa menghilangkan peran manusia sebagai pengambil keputusan utama.


\section{Merangkum Catatan dan Observasi Harian}

Dalam kegiatan belajar-mengajar, guru sering kali mencatat berbagai hal penting seperti dinamika kelas, perilaku siswa, hasil diskusi kelompok, serta refleksi harian terhadap proses pembelajaran. Selain itu, banyak pula rapat atau pertemuan informal yang menghasilkan keputusan penting, namun tidak selalu terdokumentasi dengan baik. Jika tidak segera dirangkum, catatan-catatan ini bisa tercecer atau terlupakan, padahal sangat bermanfaat untuk evaluasi dan perencanaan pembelajaran berikutnya.

Dengan bantuan Large Language Model (LLM) seperti ChatGPT, guru dapat merangkum catatan harian dan observasi secara cepat dan sistematis. Cukup dengan memberikan daftar poin atau catatan singkat, LLM dapat menyusun ringkasan dalam bentuk narasi yang rapi, laporan reflektif, atau notulensi yang siap digunakan sebagai dokumentasi resmi.

\textbf{Jenis catatan yang dapat dirangkum menggunakan LLM:}
\begin{itemize}
	\item Observasi perilaku atau kemajuan belajar siswa
	\item Catatan reflektif guru setelah pelaksanaan pembelajaran
	\item Hasil diskusi kelompok atau kegiatan proyek siswa
	\item Poin-poin penting dalam pertemuan tim guru atau MGMP
	\item Tanggapan lisan siswa dalam diskusi kelas
\end{itemize}

\textbf{Contoh prompt untuk merangkum observasi harian siswa:}

\begin{quote}
	\centering
	\texttt{"Rangkum catatan observasi berikut menjadi paragraf reflektif: siswa aktif bertanya, kurang fokus di awal pelajaran, tertarik saat eksperimen, bekerja sama dengan baik dalam kelompok, perlu didampingi saat menjawab soal tertulis."}
\end{quote}

\textbf{Hasil yang mungkin diberikan LLM:}

\begin{quote}
	Hari ini siswa menunjukkan ketertarikan yang tinggi terutama saat sesi eksperimen. Ia tampak aktif bertanya dan antusias berdiskusi dalam kelompok. Meski pada awal pembelajaran terlihat kurang fokus, perhatian siswa meningkat ketika diberikan aktivitas praktis. Dalam pengerjaan soal tertulis, siswa masih memerlukan pendampingan, namun secara umum telah menunjukkan kerja sama yang baik dan rasa ingin tahu yang tinggi.
\end{quote}

\textbf{Contoh prompt untuk merangkum hasil diskusi guru:}

\begin{quote}
	\centering
	\texttt{"Buatkan ringkasan notulensi dari poin berikut: evaluasi hasil UTS, siswa kesulitan soal analisis, rencana pendampingan khusus minggu depan, usulan ujian praktik untuk penilaian akhir semester."}
\end{quote}

\textbf{Keuntungan merangkum catatan menggunakan LLM:}
\begin{itemize}
	\item Mengubah catatan mentah menjadi dokumentasi yang tertata dan mudah dibaca
	\item Menghemat waktu dalam membuat laporan informal atau reflektif
	\item Membantu guru menyusun arsip perkembangan siswa dari waktu ke waktu
	\item Mendukung praktik refleksi profesional dan pelaporan yang transparan
\end{itemize}

LLM juga memungkinkan guru membuat versi ringkasan yang disesuaikan—misalnya dalam bentuk narasi untuk laporan, atau daftar poin untuk rekap internal. Dengan pendekatan ini, guru tetap memegang kendali atas isi dan konteks, sementara AI membantu mempercepat proses administratif tanpa kehilangan nilai pedagogis.

Pemanfaatan LLM untuk merangkum catatan harian menjadi salah satu contoh nyata bagaimana teknologi dapat mendukung praktik reflektif dan dokumentatif yang berkelanjutan di dunia pendidikan.


\section{Latihan Menyusun Komunikasi dan Dokumen Sekolah}

Setelah memahami bagaimana LLM dapat digunakan untuk menyusun berbagai bentuk komunikasi dan dokumen administratif, sesi ini dirancang untuk memberikan pengalaman langsung dalam membuat surat, email, dan laporan sekolah berdasarkan masukan teks yang sederhana. Latihan ini bertujuan untuk mengembangkan keterampilan praktis peserta dalam merancang pesan profesional dengan cepat dan efisien, sekaligus memahami batasan dan potensi penggunaan AI dalam konteks administratif pendidikan.

\textbf{Tujuan latihan:}
\begin{itemize}
	\item Menggunakan LLM untuk membuat surat resmi sekolah
	\item Menyusun email pemberitahuan kepada orang tua siswa
	\item Membuat laporan kegiatan atau observasi dari catatan harian
	\item Menyesuaikan gaya bahasa dokumen sesuai kebutuhan penerima
\end{itemize}

\textbf{Langkah-langkah latihan:}

\textbf{1. Menyusun Surat Tugas atau Undangan Resmi}

Peserta diberikan skenario administratif, misalnya penugasan guru ke pelatihan atau undangan rapat orang tua. Buat prompt LLM seperti:

\begin{quote}
	\centering
	\texttt{"Buatkan surat undangan resmi kepada orang tua siswa kelas 6 untuk menghadiri pertemuan kelas pada hari Jumat pukul 14.00 di ruang serbaguna. Gunakan gaya bahasa formal."}
\end{quote}

\textbf{2. Menulis Email Pemberitahuan Kegiatan Sekolah}

Gunakan LLM untuk menyusun email yang ditujukan kepada orang tua atau guru lain terkait kegiatan sekolah. Contoh prompt:

\begin{quote}
	\centering
	\texttt{"Tulis email kepada orang tua siswa mengenai pelaksanaan kegiatan Pramuka yang akan diadakan Sabtu depan. Sertakan jadwal dan perlengkapan yang harus dibawa."}
\end{quote}

\textbf{3. Membuat Laporan dari Catatan Observasi atau Diskusi}

Berikan catatan sederhana seperti hasil pengamatan terhadap siswa atau kesimpulan rapat, dan gunakan LLM untuk mengubahnya menjadi laporan naratif. Contoh prompt:

\begin{quote}
	\centering
	\texttt{"Susun laporan ringkas berdasarkan poin berikut: siswa aktif berdiskusi, menyelesaikan proyek tepat waktu, menunjukkan kreativitas tinggi, perlu pendampingan dalam presentasi lisan."}
\end{quote}

\textbf{4. Uji Coba Penyesuaian Nada dan Format}

Setelah memperoleh hasil dari LLM, peserta diminta mengubah nada tulisan, misalnya menjadi lebih ramah, lebih tegas, atau lebih ringkas, tergantung pada audiens yang dituju. Diskusikan:
\begin{itemize}
	\item Bagaimana nada formal berbeda dengan nada santai?
	\item Apakah struktur surat sudah sesuai dengan standar sekolah?
	\item Bagaimana memastikan kejelasan dan sopan santun dalam komunikasi tertulis?
\end{itemize}

\textbf{5. Refleksi dan Evaluasi}

Ajak peserta meninjau hasil dokumen yang telah disusun dengan LLM:
\begin{itemize}
	\item Apa yang sudah efektif?
	\item Apa yang perlu diperbaiki secara manual?
	\item Sejauh mana LLM bisa membantu dalam tugas administratif sehari-hari?
\end{itemize}

\textbf{Manfaat dari latihan ini:}
\begin{itemize}
	\item Meningkatkan keterampilan digital dan administratif guru
	\item Menunjukkan bagaimana teknologi dapat mempercepat pekerjaan tanpa mengurangi kualitas komunikasi
	\item Memberikan pemahaman langsung tentang bagaimana memanfaatkan LLM secara etis dan produktif
\end{itemize}

Latihan ini menjadi langkah awal dalam memanfaatkan AI sebagai asisten administratif, bukan hanya untuk guru, tetapi juga untuk pengelola sekolah yang membutuhkan dokumentasi cepat, akurat, dan tetap profesional.

\section*{Latihan Praktik: Dokumen dan Komunikasi Otomatis}
\addcontentsline{toc}{section}{Latihan Praktik: Dokumen dan Komunikasi Otomatis}
\begin{itemize}
	\item \textbf{Tujuan:} Membuat dokumen komunikasi dan laporan dengan cepat dan tepat menggunakan LLM.
	\item \textbf{Tugas:}
	\begin{itemize}
		\item Buat surat pemberitahuan kegiatan kelas kepada orang tua.
		\item Susun laporan perkembangan belajar berdasarkan poin-poin hasil observasi siswa.
		\item Gunakan LLM untuk menuliskan notulensi rapat dari daftar topik yang telah dibahas.
	\end{itemize}
\end{itemize}
