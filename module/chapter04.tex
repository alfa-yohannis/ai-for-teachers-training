\chapter{Riset dan Ringkasan Konten dengan Bantuan AI}

\section{Mengapa Ringkasan dan Pencarian Konten Penting dalam Pembelajaran}

Dalam era digital saat ini, informasi tersedia dalam jumlah yang sangat besar dan dalam berbagai bentuk—mulai dari artikel ilmiah, berita, laporan penelitian, hingga video pembelajaran. Meskipun hal ini memberikan kemudahan akses, banyak pendidik dan peserta didik justru menghadapi tantangan baru: bagaimana menemukan informasi yang relevan secara cepat, dan bagaimana menyederhanakan konten yang kompleks agar dapat dipahami dengan mudah.

Proses mencari dan memahami konten pendidikan sering kali memakan waktu yang tidak sedikit. Guru harus menelusuri berbagai sumber untuk menemukan materi yang sesuai kurikulum dan tingkat pemahaman siswa. Di sisi lain, siswa dapat merasa kewalahan oleh panjangnya teks atau kedalaman materi yang tidak disesuaikan dengan usia dan kemampuan mereka. Kondisi ini dapat menghambat proses belajar mengajar yang seharusnya berjalan efisien dan menyenangkan.

Large Language Model (LLM) seperti ChatGPT hadir sebagai solusi untuk menjawab tantangan tersebut. Dengan memberikan perintah yang tepat, LLM dapat membantu merangkum artikel panjang menjadi beberapa poin utama yang mudah dipahami. Ini sangat berguna untuk:
\begin{itemize}
	\item Menyederhanakan artikel atau bacaan ilmiah menjadi versi ringkas untuk siswa.
	\item Mencari inti sari dari materi sebelum disampaikan di kelas.
	\item Membantu guru menyusun materi baru dari berbagai referensi.
\end{itemize}

Lebih dari sekadar alat pencari, LLM juga dapat digunakan untuk menghasilkan konten pengajaran baru secara otomatis berdasarkan topik yang diinginkan. Dengan kemampuannya merangkum dan menyusun ulang informasi, AI memungkinkan guru bekerja lebih efisien dan tetap menjaga kualitas pembelajaran yang kontekstual serta sesuai kebutuhan peserta didik.

Oleh karena itu, keterampilan dalam memanfaatkan AI untuk ringkasan dan pencarian konten bukan hanya mendukung produktivitas, tetapi juga berkontribusi pada peningkatan kualitas proses belajar yang adaptif, relevan, dan berbasis pemahaman.

\section{Menggunakan LLM untuk Meringkas Artikel dan Teks Panjang}

Membaca dan memahami artikel panjang, terutama yang bersifat ilmiah atau informatif, sering kali menjadi tantangan bagi guru dan siswa. Terlebih ketika waktu terbatas, namun pemahaman terhadap isi teks sangat dibutuhkan untuk mendukung proses pembelajaran. Dalam konteks ini, Large Language Model (LLM) seperti ChatGPT dapat dimanfaatkan untuk menghasilkan ringkasan otomatis yang akurat, ringkas, dan mudah dipahami.

LLM dapat merangkum berbagai jenis teks, mulai dari artikel jurnal ilmiah, berita terkini, laporan hasil penelitian, hingga tulisan populer seperti blog edukasi atau artikel opini. Ringkasan ini dapat digunakan oleh guru untuk:
\begin{itemize}
	\item Menyederhanakan bahan ajar agar lebih mudah dipahami oleh siswa.
	\item Menyediakan versi ringkas dari sumber rujukan untuk diskusi kelas.
	\item Menghemat waktu dalam memahami banyak bacaan sekaligus.
\end{itemize}

Contoh prompt untuk merangkum artikel:
\begin{quote}\centering
	\texttt{"Ringkas artikel berikut menjadi maksimal 5 kalimat yang mudah dipahami siswa SMA: [salin isi artikel]."}
\end{quote}

Selain ringkasan standar, LLM juga dapat diinstruksikan untuk menyusun versi tertentu dari ringkasan, seperti:
\begin{itemize}
	\item \textbf{Ringkasan tematik} – hanya menyertakan informasi terkait topik tertentu (misal: dampak perubahan iklim).
	\item \textbf{Ringkasan berdasarkan tingkat pendidikan} – disesuaikan untuk siswa SD, SMP, atau SMA.
	\item \textbf{Ringkasan dalam bentuk poin} – mempermudah siswa dalam mencatat atau meninjau kembali isi artikel.
\end{itemize}

Sebagai ilustrasi, jika diberikan artikel sepanjang 1.000 kata tentang krisis air bersih global, LLM dapat menghasilkan ringkasan dalam bentuk berikut:
\begin{quote}\centering
	\texttt{"Krisis air bersih semakin memburuk akibat pertumbuhan populasi dan perubahan iklim. Banyak negara kesulitan memenuhi kebutuhan air warganya. Solusi seperti teknologi desalinasi dan konservasi air sedang dikembangkan. Akses air bersih menjadi salah satu target utama Tujuan Pembangunan Berkelanjutan (SDGs). Pendidikan dan kesadaran masyarakat juga memegang peran penting."}
\end{quote}

Namun, penting untuk diingat bahwa hasil ringkasan tetap perlu ditinjau oleh pendidik. Ini bertujuan untuk memastikan bahwa isi ringkasan akurat, sesuai dengan konteks pembelajaran, dan tidak melewatkan informasi penting. Dengan pemanfaatan yang tepat, LLM menjadi alat bantu yang sangat berguna untuk menyederhanakan informasi kompleks dan mempercepat pemahaman dalam proses belajar.

\section{Mencari dan Membuat Sumber Belajar Baru}

Menyusun sumber belajar yang sesuai dengan kebutuhan kelas merupakan bagian penting dari proses mengajar. Namun, keterbatasan waktu, banyaknya topik yang harus disampaikan, serta variasi kebutuhan siswa membuat guru seringkali kesulitan dalam menemukan atau membuat materi ajar yang benar-benar relevan dan efektif. Dalam konteks inilah Large Language Model (LLM) seperti ChatGPT dapat dimanfaatkan secara optimal.

LLM dapat membantu dalam dua cara utama: 
(1) mencari dan menyarankan sumber belajar yang sudah tersedia di internet, dan  
(2) membuat materi ajar baru berdasarkan topik atau kompetensi tertentu yang diinginkan oleh guru.

\textbf{1. Mencari Sumber Belajar yang Relevan}.  
LLM dapat membantu dengan menyarankan jenis sumber yang tepat, seperti video pembelajaran, artikel, buku, atau simulasi daring berdasarkan kata kunci topik. Contohnya:

\begin{quote}
	\centering
	\texttt{"Rekomendasikan 5 sumber belajar daring untuk topik fotosintesis tingkat SMP."}
\end{quote}

Model akan memberikan daftar referensi dan penjelasan ringkas mengenai isi dan tingkat kesesuaian setiap sumber dengan peserta didik.

\textbf{2. Membuat Materi Pembelajaran Baru Secara Otomatis}.  
LLM juga bisa langsung digunakan untuk membuat materi baru sesuai kebutuhan kelas, misalnya:
\begin{itemize}
	\item Penjelasan konsep dalam bahasa sederhana
	\item Ilustrasi naratif untuk menjelaskan proses
	\item Teks bacaan tematik
	\item Kegiatan eksperimen sederhana atau proyek mini
\end{itemize}

Contoh prompt:

\begin{quote}
	\centering
	\texttt{"Buatkan teks bacaan informatif sepanjang 150 kata untuk siswa kelas 4 SD tentang siklus air, sertakan pertanyaan reflektif di akhir."}
\end{quote}

\textbf{3. Menyesuaikan dengan Kurikulum dan Konteks Lokal}.  
Guru juga dapat menambahkan instruksi pada prompt agar materi disesuaikan dengan kurikulum tertentu (seperti Kurikulum Merdeka), konteks budaya lokal, atau kebutuhan kelompok belajar tertentu. Misalnya:

\begin{quote}
	\centering
	\texttt{"Buatkan materi pembelajaran tematik untuk siswa kelas 3 SD tentang transportasi di daerah pedesaan, dengan contoh dari Indonesia."}
\end{quote}

\textbf{4. Variasi Format dan Media}.  
LLM tidak hanya mampu menghasilkan teks, tapi juga bisa membantu menyusun ide untuk media pembelajaran lain seperti:
\begin{itemize}
	\item Dialog percakapan antar tokoh
	\item Skenario permainan edukatif
	\item Instruksi praktikum atau aktivitas berbasis proyek
\end{itemize}

Dengan pemanfaatan yang tepat, guru dapat memiliki banyak alternatif materi ajar yang kaya, menarik, dan disesuaikan dengan kondisi kelas. Hal ini tidak hanya menghemat waktu, tetapi juga membuka ruang bagi inovasi dan diferensiasi pembelajaran yang lebih luas.

\section{Menyusun Laporan Otomatis dengan LLM}

Pembuatan laporan merupakan bagian penting dalam dokumentasi pembelajaran, baik dalam bentuk laporan hasil diskusi kelompok, observasi siswa, refleksi kegiatan belajar, hingga laporan perkembangan kelas. Sayangnya, proses penyusunan laporan sering memakan waktu dan menyita energi, terutama ketika harus menggabungkan catatan tidak terstruktur menjadi narasi yang rapi dan sistematis. Dalam konteks inilah Large Language Model (LLM) seperti ChatGPT dapat dimanfaatkan sebagai alat bantu yang sangat efektif.

LLM dapat digunakan untuk menyusun berbagai jenis laporan secara otomatis, cukup dengan memberikan masukan berupa poin-poin utama, hasil observasi, atau transkrip percakapan. Dengan prompt yang tepat, model dapat menghasilkan laporan lengkap yang menggunakan bahasa formal, terstruktur, dan siap dibagikan.

\textbf{Jenis laporan yang dapat dihasilkan antara lain:}
\begin{itemize}
	\item Laporan hasil diskusi kelompok
	\item Laporan kegiatan proyek atau kunjungan belajar
	\item Laporan observasi perkembangan siswa
	\item Laporan pertemuan orang tua dan guru
	\item Laporan refleksi kelas atau penguatan karakter
\end{itemize}

Contoh prompt:

\begin{quote}
	\centering
	\texttt{"Susun laporan hasil diskusi kelompok tentang perubahan iklim berdasarkan poin berikut: siswa memahami penyebab, dampak, dan solusi. Sertakan pendahuluan, isi utama, dan kesimpulan."}
\end{quote}

Model akan mengubah poin-poin tersebut menjadi paragraf-paragraf terstruktur yang dapat langsung dimasukkan ke dalam dokumen resmi atau laporan portofolio kelas.

Untuk laporan observasi siswa, cukup masukkan catatan singkat yang dikumpulkan selama proses pembelajaran. Contohnya:

\begin{quote}
	\centering
	\texttt{"Buat laporan observasi untuk siswa bernama Adi. Catatan: aktif saat diskusi, menyampaikan pendapat dengan percaya diri, mulai berani bertanya, kadang kurang fokus saat penjelasan guru."}
\end{quote}

LLM akan menghasilkan ringkasan yang disusun dalam format narasi formal, siap digunakan sebagai bagian dari laporan perkembangan atau sebagai bahan diskusi dengan orang tua siswa.

\textbf{Keunggulan penggunaan LLM untuk laporan otomatis:}
\begin{itemize}
	\item Menghemat waktu dalam menulis laporan rutin
	\item Membantu menyusun kalimat formal dengan cepat
	\item Mendorong dokumentasi yang lebih konsisten dan profesional
	\item Memudahkan guru yang tidak terbiasa menyusun narasi panjang
\end{itemize}

Namun demikian, meskipun laporan dihasilkan secara otomatis, tetap diperlukan penyuntingan akhir oleh guru untuk menyesuaikan konteks, gaya bahasa, dan akurasi informasi. Hasil dari LLM sebaiknya dianggap sebagai draf awal yang dapat disempurnakan sesuai kebutuhan.

Dengan bantuan AI, guru dapat lebih fokus pada analisis hasil pembelajaran dan interaksi langsung dengan siswa, tanpa harus terbebani oleh beban administratif yang berlebihan.


\section{Strategi Menyederhanakan Informasi untuk Siswa}

Dalam pembelajaran, tidak semua peserta didik memiliki kemampuan literasi yang sama. Beberapa siswa mungkin kesulitan memahami teks bacaan yang kompleks, terutama jika materi bersifat ilmiah, teknis, atau menggunakan kosakata akademik yang belum familiar. Oleh karena itu, penting bagi guru untuk memiliki strategi dalam menyederhanakan informasi, sehingga konten tetap bermakna, tetapi lebih mudah dipahami oleh seluruh siswa.

Dengan bantuan Large Language Model (LLM), proses penyederhanaan ini dapat dilakukan dengan cepat dan fleksibel. LLM memungkinkan guru untuk menghasilkan versi teks alternatif berdasarkan tingkat usia, jenjang kelas, atau kebutuhan khusus siswa, hanya dengan memodifikasi prompt.

\textbf{Beberapa strategi yang dapat digunakan untuk menyederhanakan informasi antara lain:}

\begin{itemize}
	\item \textbf{Menggunakan Bahasa yang Sederhana dan Kalimat Pendek}  
	Hindari penggunaan istilah teknis yang sulit, ubah struktur kalimat menjadi lebih langsung, dan gunakan kata-kata sehari-hari yang lebih mudah dikenali siswa.
	
	\item \textbf{Mengubah Format Teks Menjadi Poin-Poin}  
	Daftar poin dapat membantu siswa memahami informasi secara bertahap. Struktur seperti ini cocok untuk siswa dengan kesulitan konsentrasi atau membaca paragraf panjang.
	
	\item \textbf{Menggunakan Contoh Konkret dan Kontekstual}  
	Gantilah penjelasan abstrak dengan ilustrasi dari kehidupan sehari-hari yang dekat dengan dunia siswa, seperti pengalaman di rumah, sekolah, atau lingkungan sekitar.
	
	\item \textbf{Membuat Versi Naratif atau Cerita}  
	Teks dalam bentuk cerita sering kali lebih mudah dicerna. LLM dapat mengubah penjelasan ilmiah menjadi narasi dengan tokoh, latar, dan alur yang membuat siswa lebih tertarik.
	
	\item \textbf{Menyesuaikan Gaya Bahasa dengan Usia Siswa}  
	Gunakan nada yang ringan dan tidak formal untuk siswa SD, dan nada yang sedikit lebih akademik untuk siswa SMP atau SMA. LLM dapat mengatur ini berdasarkan instruksi.
	
\end{itemize}

Contoh prompt untuk menyederhanakan teks:

\begin{quote}
	\centering
	\texttt{"Sederhanakan paragraf ini untuk siswa kelas 6 SD dengan kalimat pendek dan kosakata umum: [masukkan teks]."}
\end{quote}

Atau untuk membuat versi naratif dari penjelasan ilmiah:

\begin{quote}
	\centering
	\texttt{"Ubah penjelasan tentang fotosintesis ini menjadi cerita fiksi anak-anak dengan tokoh tumbuhan."}
\end{quote}

\textbf{Manfaat dari strategi ini meliputi:}
\begin{itemize}
	\item Meningkatkan keterlibatan siswa dalam memahami materi
	\item Meningkatkan rasa percaya diri siswa terhadap kemampuan membaca dan memahami
	\item Memudahkan guru dalam melakukan diferensiasi pembelajaran
	\item Mendukung prinsip inklusivitas dalam kelas yang heterogen
\end{itemize}

Penting untuk tetap meninjau hasil penyederhanaan dari LLM agar konten tetap akurat, tidak kehilangan makna penting, dan sesuai dengan tujuan pembelajaran. Dengan menggabungkan kreativitas guru dan kemampuan AI, informasi yang kompleks dapat disampaikan secara ramah dan mudah diakses oleh semua peserta didik.

\section{Latihan Meringkas dan Membangun Materi Ajar Otomatis}

Setelah memahami konsep dan manfaat penggunaan LLM dalam menyederhanakan informasi dan menyusun sumber ajar, bagian ini mengajak peserta untuk langsung mencoba menerapkannya. Latihan ini dirancang untuk memperkuat keterampilan dalam menggunakan AI secara praktis untuk kebutuhan pembelajaran sehari-hari, khususnya dalam dua aspek utama: merangkum dan membangun ulang materi ajar.

\textbf{Tujuan latihan:}
\begin{itemize}
	\item Menggunakan LLM untuk merangkum artikel panjang atau materi referensi
	\item Menyusun ulang teks menjadi versi yang sesuai dengan tingkat literasi dan jenjang siswa
	\item Membangun materi pembelajaran baru dari kata kunci atau poin-poin
\end{itemize}

\textbf{Langkah-langkah latihan:}

\textbf{1. Pilih Teks Sumber atau Topik Materi}.  
Ambil satu artikel atau teks bacaan dengan topik yang sesuai dengan mata pelajaran yang diajarkan. Artikel bisa bersumber dari buku teks, portal berita pendidikan, atau bahan ajar yang sudah tersedia.

\textbf{2. Gunakan LLM untuk Merangkum}.  
Masukkan teks ke dalam LLM dengan instruksi untuk membuat ringkasan pendek. Contoh prompt:

\begin{quote}
	\centering
	\texttt{"Ringkas artikel berikut menjadi 5 kalimat untuk siswa kelas 9 SMP: [masukkan artikel]."}
\end{quote}

Bandingkan hasil ringkasan dengan versi asli. Apakah inti informasi tetap tersampaikan? Apakah sudah sesuai dengan tingkat pemahaman siswa?

\textbf{3. Susun Ulang Materi Menjadi Versi Sederhana atau Naratif}.  
Lanjutkan dengan mengubah teks menjadi bentuk yang lebih sesuai untuk gaya belajar siswa, misalnya narasi atau daftar poin. Contoh prompt:

\begin{quote}
	\centering
	\texttt{"Ubah penjelasan ini menjadi cerita pendek dengan tokoh anak-anak yang menjelaskan konsep gaya gravitasi secara tidak langsung."}
\end{quote}

Atau:

\begin{quote}
	\centering
	\texttt{"Tulis ulang teks ini dalam bentuk poin-poin agar mudah dipelajari siswa SD kelas 5."}
\end{quote}

\textbf{4. Buat Versi Alternatif Berdasarkan Level Kemampuan}.  
Mintalah LLM untuk membuat versi materi yang berbeda-beda sesuai dengan level kemampuan siswa. Misalnya:

\begin{quote}
	\centering
	\texttt{"Buat dua versi materi tentang sistem pernapasan: satu untuk siswa yang sudah paham dasar-dasarnya, dan satu untuk siswa yang baru mengenalnya."}
\end{quote}

\textbf{5. Refleksi dan Penyesuaian Manual}.  
Tinjau hasil yang dihasilkan oleh LLM. Diskusikan:
\begin{itemize}
	\item Apakah bahasa dan struktur teks sudah sesuai?
	\item Apakah semua informasi penting tetap ada?
	\item Apa yang perlu disesuaikan agar materi benar-benar siap digunakan di kelas?
\end{itemize}

\textbf{Manfaat latihan ini:}
\begin{itemize}
	\item Meningkatkan efisiensi guru dalam menyusun materi pembelajaran
	\item Menyediakan lebih banyak variasi bahan ajar yang sesuai kebutuhan siswa
	\item Mendorong pemahaman praktis dalam menggabungkan AI dan pedagogi
\end{itemize}

Latihan ini diharapkan dapat memperkuat keterampilan peserta dalam menghasilkan materi ajar yang kontekstual, terjangkau secara kognitif, dan menarik secara penyajian—dengan dukungan teknologi yang cerdas dan adaptif.

\section*{Latihan Praktik: Ringkasan, Sumber Ajar, dan Laporan Otomatis}
\addcontentsline{toc}{section}{Latihan Praktik: Ringkasan, Sumber Ajar, dan Laporan Otomatis}
\begin{itemize}
	\item \textbf{Tujuan:} Menggunakan LLM untuk menyederhanakan dan menyusun ulang informasi pendidikan.
	\item \textbf{Tugas:}
	\begin{itemize}
		\item Masukkan artikel berita atau ilmiah, lalu buat ringkasan dalam 5–7 kalimat.
		\item Gunakan LLM untuk membuat materi pembelajaran dari topik “ekosistem” untuk siswa SMP.
		\item Buat laporan otomatis dari teks hasil wawancara atau catatan observasi siswa.
	\end{itemize}
\end{itemize}
