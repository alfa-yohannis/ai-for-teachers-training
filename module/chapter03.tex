\chapter{Umpan Balik dan Penilaian Personal dengan Bantuan AI}

\section{Mengapa Penilaian yang Responsif dan Personal Penting}

Penilaian bukan hanya alat untuk mengukur hasil belajar siswa, tetapi juga merupakan bagian integral dari proses pembelajaran itu sendiri. Penilaian yang dilakukan secara responsif dan personal memberikan dampak yang jauh lebih besar dibandingkan sekadar angka akhir atau nilai ujian. Ketika siswa menerima umpan balik yang sesuai dengan kebutuhan dan kemampuannya, proses belajar menjadi lebih bermakna, reflektif, dan berkelanjutan.

\textbf{Penilaian yang responsif} berarti guru mampu menangkap perkembangan belajar siswa secara tepat waktu dan menyesuaikan strategi pembelajaran berdasarkan kondisi aktual di kelas. Misalnya, ketika seorang siswa menunjukkan kesulitan dalam memahami konsep, guru dapat segera memberikan penjelasan tambahan atau latihan berbeda. Teknologi seperti Large Language Model (LLM) dapat membantu guru merancang umpan balik yang cepat, terarah, dan tetap personal.

\textbf{Penilaian yang personal} berarti bahwa setiap siswa dipandang sebagai individu yang unik—dengan gaya belajar, tingkat kemampuan, dan kebutuhan yang berbeda-beda. Umpan balik personal tidak hanya menyampaikan apa yang benar atau salah, tetapi juga membantu siswa memahami mengapa suatu jawaban salah, bagaimana cara memperbaikinya, dan apa langkah selanjutnya yang bisa dilakukan.

\textbf{Manfaat penilaian responsif dan personal antara lain:}
\begin{itemize}
	\item Meningkatkan motivasi belajar siswa karena merasa dihargai dan diperhatikan
	\item Membantu siswa mengenali kekuatan dan kelemahannya sendiri secara lebih sadar
	\item Mendorong pembelajaran yang bersifat reflektif dan tidak hanya mengejar nilai
	\item Memberikan dasar yang kuat untuk pembelajaran diferensiasi dan intervensi dini
\end{itemize}

Dalam praktiknya, guru sering kali menghadapi keterbatasan waktu untuk memberikan umpan balik yang mendalam dan menyusun laporan perkembangan secara individual. Di sinilah teknologi berbasis AI seperti LLM menjadi relevan: dengan instruksi yang tepat, guru dapat dengan cepat menghasilkan komentar atau evaluasi tertulis yang disesuaikan dengan karakteristik jawaban siswa.

Sebagai contoh, jika siswa menulis esai atau laporan singkat, LLM dapat membantu menilai aspek bahasa, isi, dan struktur, lalu menyusun umpan balik yang jelas dan membangun. Hasil ini dapat menjadi draf awal yang kemudian disempurnakan oleh guru sesuai konteks dan gaya komunikasi yang diinginkan.

Dengan penilaian yang responsif dan personal, pembelajaran tidak hanya berfokus pada hasil akhir, tetapi juga memperhatikan proses dan perkembangan yang dialami siswa secara individual. Pendekatan ini sangat sejalan dengan prinsip pendidikan yang humanis, reflektif, dan berorientasi pada pertumbuhan jangka panjang.


\section{Menyusun Umpan Balik Otomatis untuk Tugas Siswa}

Memberikan umpan balik terhadap tugas siswa merupakan bagian penting dalam proses pembelajaran yang efektif. Umpan balik yang tepat dapat membantu siswa memahami kesalahannya, mengembangkan potensi, dan meningkatkan kinerjanya secara berkelanjutan. Namun, memberikan umpan balik yang mendalam dan personal untuk setiap siswa, terutama dalam kelas besar, sering kali menjadi tantangan bagi pendidik. Di sinilah peran teknologi, khususnya Large Language Model (LLM) seperti ChatGPT, menjadi relevan dan strategis.

LLM dapat digunakan untuk membantu menyusun umpan balik otomatis yang bersifat konstruktif, spesifik, dan disesuaikan dengan isi tugas siswa. Hanya dengan memasukkan jawaban atau teks esai siswa dan beberapa instruksi tambahan, LLM dapat menghasilkan komentar yang menilai kekuatan dan kelemahan isi, struktur, serta gaya penulisan.

\textbf{Jenis tugas yang cocok untuk umpan balik otomatis:}
\begin{itemize}
	\item Esai atau tulisan reflektif
	\item Laporan hasil eksperimen atau proyek
	\item Jawaban uraian pendek
	\item Deskripsi hasil observasi atau analisis data
\end{itemize}

Contoh prompt untuk menghasilkan umpan balik terhadap esai:

\begin{quote}
	\centering
	\texttt{"Berikan umpan balik konstruktif untuk esai berikut. Tinjau aspek isi, struktur argumen, dan penggunaan bahasa: [salin isi esai]."}
\end{quote}

Contoh hasil dari prompt tersebut dapat mencakup:
\begin{itemize}
	\item Apresiasi terhadap kekuatan tulisan (misalnya: ide utama yang jelas, penggunaan contoh yang tepat)
	\item Identifikasi area yang perlu ditingkatkan (misalnya: transisi antar paragraf kurang halus, terlalu banyak pengulangan)
	\item Saran yang spesifik untuk perbaikan (misalnya: “Coba tambahkan data atau kutipan untuk mendukung argumen pada paragraf ketiga.”)
\end{itemize}

Selain itu, LLM juga dapat menyesuaikan gaya bahasa umpan balik, seperti formal, ramah, atau instruktif—tergantung pada kebutuhan guru dan karakteristik siswa. Berikut contoh prompt untuk menyesuaikan gaya:

\begin{quote}
	\centering
	\texttt{"Tulis umpan balik ramah dan memotivasi untuk siswa SMP berdasarkan teks berikut."}
\end{quote}

Penggunaan LLM untuk umpan balik otomatis tidak hanya menghemat waktu guru, tetapi juga membuka peluang bagi siswa untuk mendapatkan komentar yang lebih kaya dan reflektif. Bahkan dalam tugas yang tidak dinilai secara numerik, umpan balik ini tetap dapat membantu siswa menyadari perkembangan dan area peningkatan diri.

\textbf{Tips untuk penggunaan LLM secara efektif dalam umpan balik:}
\begin{itemize}
	\item Gunakan prompt yang spesifik dan arahkan model pada kriteria penilaian yang relevan.
	\item Tinjau hasil LLM dan sesuaikan sesuai konteks siswa.
	\item Gunakan umpan balik sebagai dasar untuk diskusi tatap muka atau refleksi mandiri siswa.
\end{itemize}

Dengan pendekatan ini, guru dapat tetap menjaga kualitas interaksi pembelajaran yang personal, meskipun dibantu oleh teknologi. Umpan balik yang disusun secara cerdas dan bijak menjadi jembatan penting untuk membangun motivasi dan pemahaman siswa dalam jangka panjang.


\section{Membuat Rubrik Penilaian Otomatis}

Rubrik penilaian merupakan alat bantu penting bagi guru dalam mengevaluasi hasil belajar siswa secara objektif dan konsisten. Dengan rubrik, kriteria penilaian dapat dijabarkan secara jelas dan terstruktur, sehingga baik guru maupun siswa memahami apa yang diharapkan dari suatu tugas. Namun, menyusun rubrik dari awal sering kali membutuhkan waktu dan tenaga, terutama jika harus disesuaikan dengan indikator kompetensi yang beragam. Dalam konteks ini, Large Language Model (LLM) seperti ChatGPT dapat dimanfaatkan untuk menyusun rubrik penilaian secara otomatis berdasarkan kriteria yang diberikan.

LLM dapat membantu merancang rubrik untuk berbagai jenis tugas, seperti esai, proyek, presentasi, eksperimen, maupun portofolio. Cukup dengan memberikan informasi mengenai tujuan pembelajaran, jenis tugas, dan indikator yang ingin dinilai, LLM dapat menyusun rubrik dalam format tabel dengan deskriptor yang spesifik dan relevan.

\textbf{Langkah-langkah menyusun rubrik dengan LLM:}

\textbf{1. Tentukan indikator atau aspek yang ingin dinilai}  
Misalnya:
\begin{itemize}
	\item Isi dan relevansi konten
	\item Struktur dan organisasi tulisan
	\item Kemampuan analisis atau argumentasi
	\item Tata bahasa dan ejaan
\end{itemize}

\textbf{2. Gunakan prompt untuk membuat rubrik berdasarkan indikator tersebut}  
Contoh prompt:

\begin{quote}
	\centering
	\texttt{"Buatkan rubrik penilaian esai untuk siswa SMA. Aspek yang dinilai: isi, struktur, penggunaan bahasa, dan orisinalitas. Tampilkan dalam format 4 level (Sangat Baik, Baik, Cukup, Perlu Perbaikan)."}
\end{quote}

\textbf{3. Tinjau hasil rubrik dan sesuaikan jika perlu}  
Hasil dari LLM biasanya langsung disajikan dalam format tabel atau daftar deskriptif. Namun, guru tetap perlu menyesuaikan bahasa dan bobot sesuai kebutuhan kurikulum atau karakteristik siswa.

\textbf{Contoh hasil rubrik otomatis (ringkas):}

\begin{table}
	\centering
	\renewcommand{\arraystretch}{1.4}
	\begin{tabularx}{\textwidth}{|l|X|X|X|X|}
		\hline
		\textbf{Aspek} & \textbf{Sangat Baik} & \textbf{Baik} & \textbf{Cukup} & \textbf{Perlu Perbaikan} \\
		\hline
		\textbf{Isi} & Isi lengkap, mendalam, relevan & Isi cukup lengkap dan relevan & Isi kurang lengkap & Isi tidak sesuai topik \\
		\hline
		\textbf{Struktur} & Paragraf tersusun rapi, logis & Struktur cukup jelas & Struktur agak membingungkan & Tidak ada struktur yang jelas \\
		\hline
		\textbf{Bahasa} & Bahasa baku, bebas kesalahan & Sedikit kesalahan tata bahasa & Banyak kesalahan tata bahasa & Sulit dipahami \\
		\hline
		\textbf{Orisinalitas} & Gagasan orisinal, analisis kuat & Cukup orisinal, analisis cukup & Kurang orisinal & Cenderung menyalin sumber lain \\
		\hline
	\end{tabularx}
	\caption{Contoh Rubrik Penilaian Esai Otomatis}
	\label{tab:rubrik-esai}
\end{table}

\textbf{4. Buat versi yang disesuaikan dengan jenjang dan kebutuhan belajar}  
Prompt dapat dimodifikasi untuk membuat rubrik yang lebih sederhana untuk jenjang SD atau lebih kompleks untuk SMA. Misalnya:

\begin{quote}
	\centering
	\texttt{"Buat rubrik penilaian untuk presentasi siswa kelas 5 SD dengan tiga aspek penilaian: kejelasan penyampaian, penggunaan gambar, dan kerja sama kelompok."}
\end{quote}

\textbf{Manfaat membuat rubrik otomatis dengan LLM:}
\begin{itemize}
	\item Menghemat waktu dalam menyusun instrumen penilaian yang konsisten
	\item Memudahkan diferensiasi dan adaptasi sesuai mata pelajaran atau jenjang
	\item Meningkatkan transparansi penilaian karena siswa dapat melihat harapan secara eksplisit
	\item Mendorong pembelajaran berbasis tujuan dan refleksi
\end{itemize}

Dengan LLM, guru tidak lagi harus memulai dari nol setiap kali menyusun rubrik, namun tetap memiliki kendali untuk menyempurnakan dan menyesuaikan hasilnya. Hal ini menjadikan proses penilaian lebih efisien, terstruktur, dan mendukung pembelajaran yang adil dan terarah.

\section{Membangun Alat Refleksi Diri dan Penilaian Mandiri}

Refleksi diri dan penilaian mandiri merupakan keterampilan penting yang mendukung perkembangan metakognitif siswa. Ketika siswa mampu menilai pemahamannya sendiri, mengenali kekuatan dan kelemahannya, serta merancang strategi belajar yang sesuai, mereka akan menjadi pembelajar yang lebih mandiri, percaya diri, dan bertanggung jawab. Namun, membimbing siswa untuk melakukan refleksi yang bermakna tidak selalu mudah. Guru memerlukan alat bantu berupa pertanyaan, panduan, atau ceklis yang relevan dan mudah digunakan.

Large Language Model (LLM) seperti ChatGPT dapat dimanfaatkan untuk menyusun berbagai bentuk alat refleksi dan evaluasi mandiri yang dapat disesuaikan dengan usia, tingkat kemampuan, dan jenis tugas siswa. Dengan hanya memberikan topik atau deskripsi tugas, LLM dapat menghasilkan daftar pertanyaan reflektif, format jurnal belajar, atau lembar evaluasi diri yang siap digunakan di kelas.

\textbf{Jenis alat refleksi dan evaluasi diri yang dapat dibuat dengan LLM:}
\begin{itemize}
	\item Daftar pertanyaan reflektif setelah menyelesaikan tugas atau proyek
	\item Ceklis pemantauan kemajuan belajar harian atau mingguan
	\item Panduan jurnal belajar untuk mendokumentasikan proses dan pemahaman
	\item Rubrik penilaian mandiri untuk membandingkan hasil kerja dengan kriteria
	\item Pertanyaan metakognitif seperti “Apa yang sudah saya pahami?” dan “Apa yang masih membingungkan?”
\end{itemize}

\textbf{Contoh prompt:}

\begin{quote}
	\centering
	\texttt{"Buatkan 5 pertanyaan refleksi diri untuk siswa SMP setelah menyelesaikan proyek sains tentang ekosistem."}
\end{quote}

\begin{quote}
	\centering
	\texttt{"Susun format penilaian mandiri untuk tugas menulis esai, berdasarkan 3 aspek: isi, struktur, dan bahasa."}
\end{quote}

\textbf{Contoh hasil pertanyaan reflektif:}
\begin{itemize}
	\item Apa bagian tersulit dari proyek ini dan bagaimana saya mengatasinya?
	\item Apakah saya bekerja dengan baik dalam kelompok? Mengapa atau mengapa tidak?
	\item Apa satu hal baru yang saya pelajari tentang ekosistem?
	\item Bagaimana saya bisa membuat proyek ini lebih baik jika saya mengulanginya?
	\item Apakah saya bangga dengan hasil kerja saya? Jelaskan alasannya.
\end{itemize}

\textbf{Manfaat membangun alat refleksi dengan LLM:}
\begin{itemize}
	\item Membantu siswa berpikir kritis terhadap proses belajarnya sendiri
	\item Memudahkan guru menyiapkan instrumen refleksi tanpa membuat dari nol
	\item Meningkatkan keterlibatan siswa dalam proses evaluasi
	\item Mendorong pembelajaran yang lebih mandiri dan berkelanjutan
\end{itemize}

Untuk hasil yang maksimal, guru dapat mengadaptasi atau menyederhanakan hasil dari LLM agar lebih kontekstual dan sesuai dengan budaya kelas masing-masing. Refleksi dan evaluasi diri yang konsisten akan memperkuat kesadaran siswa akan proses belajarnya dan mendorong mereka untuk terus berkembang secara aktif.


\section{Contoh Kasus Penggunaan Umpan Balik dalam Konteks Kelas}

Untuk memahami penerapan nyata dari umpan balik otomatis menggunakan LLM dalam pembelajaran, mari kita telaah sebuah studi kasus mini yang menggambarkan bagaimana teknologi ini digunakan di kelas, serta dampak yang dihasilkan terhadap pengalaman belajar siswa.

\textbf{Studi Kasus: Kelas Bahasa Indonesia SMA – Menulis Esai Argumentatif}

Di sebuah kelas Bahasa Indonesia tingkat SMA, guru meminta siswa untuk menulis esai argumentatif dengan topik: "Apakah media sosial membawa lebih banyak manfaat atau kerugian bagi remaja?" Terdapat 36 siswa yang mengumpulkan tugas secara daring melalui platform pembelajaran.

\textbf{Tantangan:}
Guru ingin memberikan umpan balik mendalam untuk setiap esai—meliputi aspek struktur, kekuatan argumen, dan penggunaan bahasa. Namun, dengan keterbatasan waktu, memberi komentar satu per satu secara manual akan sangat memakan waktu dan bisa mengurangi konsistensi penilaian.

\textbf{Solusi: Menggunakan LLM untuk Membantu Umpan Balik Otomatis}

Guru kemudian menggunakan LLM untuk menghasilkan komentar otomatis dari setiap esai. Ia menyalin teks esai ke dalam prompt berikut:

\begin{quote}
	\centering
	\texttt{"Berikan umpan balik konstruktif dan ramah untuk esai ini. Tinjau struktur argumen, kekuatan alasan, dan penggunaan bahasa: [isi esai]."}
\end{quote}

Model kemudian menghasilkan umpan balik personal seperti:

\begin{quote}
	\itshape
	“Esai ini memiliki ide utama yang jelas dan argumen yang cukup meyakinkan. Akan lebih baik jika Anda menyertakan contoh konkret untuk memperkuat posisi Anda. Beberapa kalimat bisa disusun ulang agar lebih efektif. Secara keseluruhan, Anda menunjukkan kemampuan berpikir kritis yang baik.”
\end{quote}

\textbf{Hasil:}
\begin{itemize}
	\item Semua siswa menerima umpan balik dalam waktu singkat (kurang dari 1 hari).
	\item Beberapa siswa merasa lebih termotivasi untuk merevisi tulisannya karena komentar yang disampaikan terasa personal dan membangun.
	\item Guru tetap meninjau dan menyunting sebagian hasil, namun prosesnya jauh lebih efisien.
	\item Proses refleksi siswa meningkat: lebih dari separuh siswa menuliskan tanggapan terhadap umpan balik dalam jurnal belajar mereka.
\end{itemize}

\textbf{Refleksi Guru:}
Guru menyadari bahwa penggunaan LLM dalam memberi umpan balik bukanlah pengganti interaksi guru-siswa, melainkan pelengkap. Dengan adanya teknologi ini, ia dapat mengalokasikan lebih banyak waktu untuk diskusi kelas dan pembimbingan individu, tanpa mengorbankan kualitas evaluasi tertulis.

\textbf{Kesimpulan:}
Studi kasus ini menunjukkan bahwa pemanfaatan LLM untuk umpan balik otomatis dapat:
\begin{itemize}
	\item Meningkatkan efisiensi dan jangkauan penilaian
	\item Menumbuhkan budaya reflektif dalam menulis
	\item Memperkuat interaksi yang lebih bermakna antara guru dan siswa
\end{itemize}

Penerapan ini menunjukkan bahwa dengan strategi yang tepat, AI dapat menjadi mitra aktif dalam mendorong pembelajaran yang personal, responsif, dan berkelanjutan.

\section{Latihan Mendesain Umpan Balik dan Rubrik dengan LLM}

Setelah memahami konsep dasar dan contoh penerapan umpan balik otomatis dan rubrik penilaian, bagian ini dirancang sebagai sesi praktik langsung. Tujuannya adalah agar peserta dapat mengalami sendiri bagaimana LLM dapat digunakan untuk menyusun komentar penilaian dan rubrik secara efisien, responsif, dan kontekstual.

Latihan ini bertujuan tidak hanya untuk memperkenalkan teknis penggunaan LLM, tetapi juga untuk mengasah sensitivitas pedagogis dalam menyesuaikan hasil AI dengan karakteristik siswa dan mata pelajaran yang diajarkan.\\

\textbf{Tujuan Latihan:}
\begin{itemize}
	\item Menggunakan LLM untuk menilai jawaban atau tugas siswa
	\item Menyusun umpan balik otomatis berdasarkan kriteria tertentu
	\item Membuat rubrik penilaian yang sesuai dengan capaian pembelajaran
	\item Merefleksikan efektivitas hasil AI dalam konteks pembelajaran nyata\\
\end{itemize}

\textbf{Langkah-langkah Latihan:}\\

\textbf{1. Pilih Contoh Tugas Siswa}  
Gunakan salah satu contoh tugas seperti esai pendek, jawaban uraian, atau laporan proyek dari siswa (bisa fiktif). Alternatifnya, peserta dapat membuat jawaban singkat berdasarkan instruksi tugas yang disediakan.

\textbf{2. Susun Prompt Umpan Balik Otomatis}  
Masukkan teks tugas ke dalam LLM dengan instruksi eksplisit. Contoh prompt:

\begin{quote}
	\centering
	\texttt{"Tulis umpan balik konstruktif dan ramah untuk jawaban ini. Tinjau isi, struktur, dan penggunaan bahasa: [salin jawaban siswa]."}
\end{quote}

Tinjau hasil umpan balik yang diberikan oleh LLM. Diskusikan apakah umpan balik tersebut:
\begin{itemize}
	\item Sudah spesifik dan relevan
	\item Terlalu umum atau klise
	\item Perlu penyuntingan untuk nada atau konteks
\end{itemize}

\textbf{3. Bangun Rubrik Penilaian dengan Kriteria Sederhana}  
Susun prompt untuk menghasilkan rubrik berdasarkan indikator pembelajaran. Contoh:

\begin{quote}
	\centering
	\texttt{"Buatkan rubrik penilaian esai singkat untuk siswa kelas 9 SMP. Aspek penilaian: isi, struktur, tata bahasa. Tampilkan dalam format 4 level."}
\end{quote}

Bandingkan hasil rubrik LLM dengan kebutuhan kelas nyata. Tambahkan atau sesuaikan deskriptor jika diperlukan.

\textbf{4. Ciptakan Variasi Sesuai Jenjang atau Gaya Bahasa}  
Uji coba membuat versi rubrik dan umpan balik yang berbeda untuk:
\begin{itemize}
	\item Siswa SD dengan bahasa yang lebih sederhana
	\item Siswa SMA dengan ekspektasi akademik lebih tinggi
	\item Versi formal vs versi ramah dan suportif
\end{itemize}

\textbf{5. Refleksi: Evaluasi Peran AI sebagai Mitra Penilaian}  
Diskusikan bersama:
\begin{itemize}
	\item Apakah penggunaan LLM mempermudah proses evaluasi?
	\item Apa yang perlu tetap dilakukan secara manual oleh guru?
	\item Bagaimana menjaga keadilan dan konteks saat menggunakan AI untuk menilai?
\end{itemize}

\textbf{Manfaat Latihan Ini:}
\begin{itemize}
	\item Meningkatkan keterampilan peserta dalam merancang evaluasi berbasis AI
	\item Memberikan pengalaman nyata bagaimana teknologi mendukung praktik penilaian yang efisien dan bermakna
	\item Menumbuhkan kesadaran kritis tentang batasan dan potensi AI dalam konteks pendidikan
\end{itemize}

Dengan latihan ini, diharapkan pendidik dapat menjadikan LLM sebagai mitra yang andal dalam proses penilaian—bukan sebagai pengganti, melainkan sebagai alat yang mendukung keputusan profesional yang bijak dan reflektif.

\section*{Latihan Praktik: Umpan Balik dan Rubrik Adaptif}
\addcontentsline{toc}{section}{Latihan Praktik: Umpan Balik dan Rubrik Adaptif}
\begin{itemize}
	\item \textbf{Tujuan:} Menyusun umpan balik dan rubrik yang sesuai dengan karakteristik tugas dan capaian pembelajaran.
	\item \textbf{Tugas:}
	\begin{itemize}
		\item Gunakan LLM untuk mengevaluasi contoh esai siswa dan berikan umpan balik otomatis.
		\item Buat rubrik penilaian untuk tugas presentasi atau proyek kelompok.
		\item Susun 5 pertanyaan refleksi yang bisa digunakan siswa untuk menilai kemajuan belajarnya sendiri.
	\end{itemize}
\end{itemize}
