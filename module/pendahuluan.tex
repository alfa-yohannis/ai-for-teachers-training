\chapter{Pendahuluan}

Perkembangan teknologi kecerdasan buatan (AI) telah menghadirkan perubahan signifikan da\-lam berbagai aspek kehidupan, termasuk dunia pendidikan. Salah satu inovasi yang paling menonjol dalam beberapa tahun terakhir adalah kemunculan \textit{Large Language Model} (LLM) se\-per\-ti ChatGPT, yang mampu memahami, menghasilkan, dan menyusun teks dalam bahasa alami dengan cara yang menyerupai manusia. Teknologi ini membuka peluang baru bagi guru untuk memperkaya proses pembelajaran, meningkatkan efisiensi, dan memberikan pengalaman belajar yang lebih adaptif.

Buku ini disusun sebagai panduan praktis bagi para pendidik untuk memahami dan menerapkan LLM dalam konteks kelas dan kegiatan pendidikan sehari-hari. Setiap bab dirancang untuk memberikan pemahaman konseptual, contoh penerapan nyata, serta latihan praktik yang memungkinkan guru langsung mencoba dan mengeksplorasi potensi teknologi ini.

Bab pertama memperkenalkan konsep dasar LLM, cara kerjanya secara sederhana, serta studi kasus penggunaan LLM dalam kelas. Guru akan diajak mengenal antarmuka seperti ChatGPT dan mencoba berbagai jenis \textit{prompt} untuk berinteraksi secara langsung.

Bab kedua membahas bagaimana LLM dapat digunakan dalam menyusun rencana pembelajaran dan membuat lembar kerja. Termasuk di dalamnya adalah proses pembuatan soal, kuis, dan penyesuaian materi untuk beragam kebutuhan belajar siswa.

Bab ketiga fokus pada umpan balik dan penilaian personal. Guru akan belajar merancang umpan balik otomatis, rubrik penilaian berbasis AI, serta membangun alat refleksi diri dan penilaian mandiri bagi siswa.

Bab keempat memperlihatkan bagaimana LLM dapat membantu dalam meringkas konten, mencari sumber belajar baru, serta menyusun laporan pembelajaran secara otomatis.

Bab kelima mengeksplorasi dukungan AI dalam tugas administratif dan komunikasi. Topik yang dibahas mencakup penyusunan email, laporan akademik, surat resmi, hingga pencatatan observasi harian.

Bab keenam mengangkat isu penting tentang etika penggunaan AI di dunia pendidikan. Guru diajak untuk memahami risiko seperti bias, ketergantungan, plagiarisme, serta pentingnya menjaga privasi dan keamanan data siswa.

\vspace{1em}
\noindent \textbf{Tujuan Pembelajaran}

Setelah mempelajari buku ini, pembaca diharapkan mampu:
\begin{enumerate}
	\item Memahami konsep dasar dan cara kerja LLM dalam konteks pendidikan.
	\item Menerapkan LLM untuk membantu perencanaan, penyusunan materi ajar, dan evaluasi pembelajaran.
	\item Mengoptimalkan penggunaan AI untuk mendukung tugas administratif dan komunikasi sekolah.
	\item Mengembangkan kesadaran kritis terhadap isu etika, bias, dan keamanan data dalam penggunaan teknologi AI.
	\item Mengintegrasikan AI secara bertanggung jawab sebagai mitra pengajaran dalam proses belajar mengajar.
\end{enumerate}

\vspace{1em}
\noindent \textbf{Profil Pembaca Sasaran}

Buku ini ditujukan terutama bagi para guru, pendidik, dan tenaga kependidikan yang tertarik mengadopsi teknologi AI dalam proses pembelajaran. Tidak diperlukan latar belakang teknis dalam bidang komputer, karena setiap topik disampaikan secara sederhana dan disertai contoh serta latihan yang dapat langsung diterapkan di ruang kelas. Buku ini juga relevan bagi instruktur pelatihan guru, mahasiswa program studi pendidikan, dan pengambil kebijakan yang ingin memahami potensi dan tantangan penerapan LLM di sektor pendidikan.

Melalui kombinasi penjelasan teoritis, studi kasus, dan latihan praktik di setiap bab, buku ini diharapkan mampu menjadi rujukan yang aplikatif dan inspiratif bagi para guru yang ingin mengadopsi teknologi AI secara efektif dan bertanggung jawab. Transformasi digital dalam pendidikan tidak dapat dihindari, namun dengan pemahaman dan pendekatan yang tepat, guru tetap akan memegang peran sentral dalam membentuk generasi masa depan.
