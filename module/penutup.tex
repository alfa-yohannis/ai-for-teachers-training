\chapter{Penutup}

Penggunaan Large Language Model (LLM) dalam dunia pendidikan membuka peluang baru untuk mentransformasi cara guru mengajar, merancang pembelajaran, memberikan umpan balik, hingga mengelola tugas-tugas administratif. Melalui enam bab sebelumnya, buku ini telah mengajak pembaca mengeksplorasi berbagai aplikasi praktis LLM mulai dari mendesain materi ajar, memberikan penilaian yang responsif, hingga mengelola komunikasi dan dokumen secara otomatis.

Pengenalan LLM di kelas bukan sekadar tren teknologi, tetapi bagian dari proses adaptasi terhadap kebutuhan pembelajaran yang semakin kompleks dan beragam. Dengan memahami cara kerja dan potensi LLM, guru dapat lebih percaya diri menggunakan AI sebagai mitra untuk meningkatkan kualitas pembelajaran, bukan sebagai pengganti peran edukatif mereka.

Namun, adopsi teknologi juga membawa tantangan etis dan risiko yang perlu disikapi dengan bijak. Bab tentang etika menjadi pengingat penting bahwa kecanggihan teknologi harus selalu diimbangi dengan pertimbangan nilai, tanggung jawab, dan perlindungan terhadap peserta didik.

Sebagai penutup, diharapkan pembaca tidak hanya memperoleh pemahaman teknis dan praktis tentang LLM, tetapi juga membangun refleksi kritis dan kesadaran profesional dalam penggunaannya. Teknologi hanyalah alat; dampak positifnya tergantung pada bagaimana dan untuk tujuan apa ia digunakan.

Mari terus belajar, bereksperimen, dan bertumbuh bersama AI—dengan tetap menempatkan nilai-nilai pendidikan sebagai fondasi utama dalam setiap langkah transformasi digital di kelas.

