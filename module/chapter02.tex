\chapter{Pembuatan Rencana Pembelajaran dan Lembar Kerja}

\section{Mengapa Perencanaan Pembelajaran yang Efisien Itu Penting}

Perencanaan pembelajaran merupakan fondasi utama dalam proses mengajar yang efektif. Rencana pembelajaran yang baik memungkinkan kegiatan belajar berlangsung terstruktur, terukur, dan sesuai dengan capaian kompetensi. Namun dalam praktiknya, banyak pendidik menghadapi keterbatasan waktu, sumber daya, dan keharusan untuk menyesuaikan materi dengan kebutuhan beragam peserta didik. Inilah alasan mengapa efisiensi dalam merancang pembelajaran menjadi sangat penting.

Dengan bantuan teknologi seperti Large Language Model (LLM), proses perencanaan dapat diotomatisasi sebagian. Hal ini berdampak langsung pada penghematan waktu dan energi dalam menyusun komponen pembelajaran, seperti tujuan pembelajaran, strategi penyampaian, serta soal evaluasi. Guru tidak perlu memulai dari nol setiap kali menyusun perangkat ajar, karena LLM dapat menghasilkan kerangka rencana pembelajaran hanya dengan memberikan deskripsi topik atau tujuan yang diinginkan.

Selain efisiensi waktu, pendekatan ini juga mendukung diferensiasi pembelajaran. Artinya, materi dan aktivitas belajar dapat dengan mudah dimodifikasi sesuai dengan tingkat kemampuan siswa, gaya belajar, atau kebutuhan khusus. LLM mampu menyesuaikan versi materi untuk siswa pemula, menengah, hingga lanjutan—bahkan menyusun variasi kegiatan yang sesuai untuk kelas yang heterogen.

Oleh karena itu, penggunaan alat bantu berbasis AI seperti LLM tidak hanya menjawab tantangan keterbatasan waktu, tetapi juga membantu guru dalam merancang pembelajaran yang lebih inklusif, adaptif, dan responsif terhadap keragaman peserta didik. Pendekatan ini memungkinkan lebih banyak waktu dialokasikan untuk interaksi bermakna di kelas dan pembimbingan individu, yang merupakan inti dari pendidikan yang berkualitas.


\section{Menyusun Rencana Pembelajaran Menggunakan LLM}

Large Language Model (LLM) seperti ChatGPT dapat membantu menyusun rencana pembelajaran secara cepat dan fleksibel. Dengan hanya memberikan deskripsi topik, jenjang pendidikan, serta tujuan pembelajaran, model dapat menghasilkan struktur rencana yang mencakup komponen utama seperti tujuan, langkah kegiatan, media, dan metode penilaian. Hal ini sangat berguna bagi pendidik yang membutuhkan inspirasi awal, kerangka kerja, atau variasi pendekatan dalam menyampaikan materi.

Sebagai contoh, berikut prompt yang dapat digunakan:
\begin{quote}
	\centering
		\texttt{“Buatkan rencana pembelajaran berdurasi 1 jam untuk siswa SMA kelas X tentang sistem pernapasan manusia, lengkap dengan tujuan pembelajaran, kegiatan inti, dan evaluasi.”}
\end{quote}

Hasil dari prompt tersebut biasanya mencakup:
\begin{itemize}
	\item \textbf{Tujuan Pembelajaran:} Siswa mampu menjelaskan proses pernapasan dan mengidentifikasi organ yang terlibat.
	\item \textbf{Pendahuluan:} Guru menanyakan pertanyaan pemantik dan memutar video pendek.
	\item \textbf{Kegiatan Inti:} Diskusi kelompok berdasarkan teks bacaan dan presentasi hasil diskusi.
	\item \textbf{Penutup:} Refleksi singkat dan kuis 5 soal pilihan ganda.
\end{itemize}

Rencana ini dapat diedit langsung oleh pendidik sesuai konteks kelas, termasuk menambahkan kegiatan berbasis proyek, menyisipkan asesmen formatif, atau menyusun alternatif aktivitas untuk siswa dengan kebutuhan khusus. Misalnya, jika terdapat siswa dengan hambatan penglihatan, bagian visualisasi dapat diganti dengan narasi atau audio.

LLM juga dapat dimanfaatkan untuk membuat versi rencana pembelajaran dalam bahasa Inggris, atau untuk kelas bilingual. Dengan menyisipkan konteks tambahan seperti gaya mengajar, durasi pembelajaran, atau pendekatan pedagogis tertentu (misalnya pembelajaran berbasis masalah atau pembelajaran tematik), hasil rencana yang dihasilkan bisa lebih personal dan kontekstual.

Penting untuk diingat bahwa hasil dari LLM bersifat usulan awal. Tinjauan dan penyesuaian tetap perlu dilakukan agar rencana pembelajaran sesuai dengan kurikulum yang berlaku, kebutuhan peserta didik, serta karakteristik kelas. Dengan pendekatan kolaboratif antara manusia dan AI, proses perencanaan menjadi lebih cepat tanpa mengorbankan kualitas.

\section{Membuat Soal dan Kuis secara Otomatis}

Pembuatan soal dan kuis merupakan salah satu kegiatan rutin yang memerlukan waktu dan ketelitian. Guru perlu memastikan bahwa soal yang disusun relevan dengan tujuan pembelajaran, memiliki tingkat kesulitan yang sesuai, serta mampu mengukur pemahaman peserta didik secara adil dan terstruktur. Dengan bantuan Large Language Model (LLM), proses ini dapat dilakukan secara otomatis dan efisien, tanpa mengorbankan variasi dan kualitas soal.

LLM seperti ChatGPT dapat digunakan untuk menghasilkan berbagai jenis soal, antara lain:
\begin{itemize}
	\item \textbf{Pilihan Ganda} – Cocok untuk asesmen cepat dan diagnosis awal pemahaman konsep.
	\item \textbf{Isian Singkat} – Mengukur kemampuan mengingat dan menerapkan informasi faktual.
	\item \textbf{Uraian Pendek} – Mendorong siswa menjelaskan proses, membandingkan konsep, atau memberikan pendapat.
	\item \textbf{Soal Benar/Salah dan Menjodohkan} – Berguna untuk latihan ringan dan kuis interaktif.
\end{itemize}

Contoh prompt yang dapat digunakan:
\begin{quote}
	\centering
	\texttt{“Buatkan 5 soal pilihan ganda tentang revolusi industri untuk siswa kelas XI IPS lengkap dengan opsi jawaban dan penjelasan jawaban yang benar.”}
\end{quote}

Hasil yang diberikan oleh LLM umumnya sudah cukup siap digunakan, lengkap dengan:
\begin{itemize}
	\item Nomor soal dan teks pertanyaan
	\item Empat pilihan jawaban (A–D)
	\item Indikasi jawaban yang benar
	\item Penjelasan singkat mengapa jawaban tersebut benar
\end{itemize}

Soal-soal ini dapat langsung dicetak atau dimasukkan ke dalam platform pembelajaran daring (seperti Google Forms, Moodle, atau Kahoot). Jika diperlukan, prompt dapat diubah agar LLM menyesuaikan gaya bahasa, tingkat kesulitan, atau tipe soal tertentu (misalnya “buat soal HOTS” atau “buat soal dengan konteks kehidupan sehari-hari”).

Selain untuk evaluasi sumatif, LLM juga bisa digunakan untuk membuat soal latihan harian, tugas remedial, maupun latihan mandiri yang berbeda-beda untuk setiap kelompok siswa. Hal ini memungkinkan pendekatan diferensiasi secara lebih praktis.

Namun, penting untuk tetap memeriksa dan menyunting hasil soal yang dihasilkan. Meskipun akurat secara umum, kadang ditemukan pertanyaan ambigu, opsi jawaban yang tidak setara, atau kekeliruan konsep kecil yang perlu diperbaiki. Dengan menyandingkan kecepatan LLM dan kepekaan pedagogis guru, pembuatan soal menjadi lebih cepat, bervariasi, dan tetap relevan secara pedagogis.


\section{Menyesuaikan Lembar Kerja untuk Beragam Kebutuhan Belajar}

Setiap kelas terdiri dari peserta didik dengan latar belakang, kemampuan, dan gaya belajar yang beragam. Oleh karena itu, lembar kerja yang digunakan dalam kegiatan belajar perlu disesuaikan agar semua siswa dapat mengakses materi secara optimal. Penyesuaian ini mencakup tidak hanya tingkat kesulitan soal, tetapi juga format penyajian, bahasa yang digunakan, dan konteks yang relevan bagi peserta didik.

Dengan bantuan Large Language Model (LLM), penyesuaian lembar kerja dapat dilakukan dengan cepat dan fleksibel. Guru cukup memberikan instruksi tambahan pada prompt untuk menyesuaikan konten, misalnya: “buat versi sederhana”, “gunakan kalimat pendek”, atau “ubah ke gaya narasi cerita untuk siswa SD”. Hasil yang dihasilkan oleh model dapat menjadi titik awal dalam merancang lembar kerja yang inklusif.

Beberapa bentuk penyesuaian yang dapat dilakukan antara lain:

\begin{itemize}
	\item \textbf{Tingkat Kesulitan:} Lembar kerja untuk siswa tingkat dasar dapat dibuat dengan pertanyaan langsung dan contoh konkret, sementara untuk siswa tingkat lanjut dapat mencakup analisis, perbandingan, atau eksplorasi ide terbuka.
	\item \textbf{Bahasa dan Format:} Gunakan kalimat sederhana, paragraf pendek, dan visual pendukung untuk siswa dengan kebutuhan khusus atau keterbatasan literasi. LLM dapat membantu menyusun versi teks yang lebih mudah dibaca tanpa kehilangan makna inti.
	\item \textbf{Kebutuhan Khusus:} Untuk siswa dengan hambatan belajar, seperti disleksia atau kesulitan fokus, LLM dapat menyusun teks dalam format bullet-point, memperbanyak ruang kosong antarbaris, atau menyertakan petunjuk dalam bentuk simbol dan warna.
	\item \textbf{Gaya Belajar:} Siswa dengan gaya belajar visual dapat dibantu dengan lembar kerja berbasis gambar, diagram, atau ilustrasi. Untuk gaya auditori, LLM dapat membantu membuat transkrip audio atau narasi yang mendampingi isi lembar kerja.
\end{itemize}

Contoh prompt yang dapat digunakan:
\begin{quote}
	\centering
	\texttt{“Buatkan lembar kerja tentang daur air untuk siswa kelas 4 SD, dengan teks pendek, gambar pendukung, dan soal pilihan ganda yang mudah dimengerti.”}
\end{quote}

Selain itu, LLM juga dapat membantu membuat beberapa versi lembar kerja untuk topik yang sama—misalnya satu untuk kelompok cepat, satu untuk kelompok sedang, dan satu untuk kelompok dengan kebutuhan pembelajaran tambahan. Ini memungkinkan guru menerapkan strategi pembelajaran terdiferensiasi dengan lebih mudah dan konsisten.

Dengan demikian, penggunaan LLM tidak hanya membantu mempercepat pembuatan lembar kerja, tetapi juga memastikan bahwa materi yang diberikan dapat diakses oleh semua siswa, tanpa memandang perbedaan kemampuan atau gaya belajarnya. Pendekatan ini memperkuat prinsip inklusivitas dalam pembelajaran.


\section{Latihan Membuat Lembar Kerja dan Kuis}

Setelah mempelajari berbagai potensi penggunaan LLM untuk perencanaan pembelajaran, bagian ini memberikan kesempatan untuk praktik langsung. Tujuannya adalah agar setiap pendidik tidak hanya memahami konsep, tetapi juga mampu menerapkannya secara mandiri dalam kegiatan belajar-mengajar.

Latihan ini berfokus pada pembuatan lembar kerja dan kuis dengan menggunakan LLM seperti ChatGPT. Setiap bagian latihan dirancang untuk menuntun secara bertahap, mulai dari perencanaan topik hingga evaluasi hasil.

\textbf{1. Tentukan Topik dan Tujuan Pembelajaran.}  
Pilih salah satu topik dari mata pelajaran yang diajarkan. Tetapkan tujuan pembelajaran secara jelas dan spesifik. Contoh:
\begin{quote}
	\centering
	\texttt{“Siswa dapat menjelaskan proses daur air dan mengidentifikasi tahapan-tahapannya.”}
\end{quote}

\textbf{2. Gunakan Prompt untuk Membuat Lembar Kerja.}  
Masukkan perintah (prompt) ke dalam LLM, seperti:
\begin{quote}
	\centering
	\texttt{“Buatkan lembar kerja tentang daur air untuk siswa kelas 5 SD, gunakan kalimat sederhana, sertakan 3 soal pilihan ganda dan 1 soal uraian pendek.”}
\end{quote}
Bandingkan hasilnya dengan kebutuhan kelas. Ubah jika perlu, lalu buat dua versi: satu untuk siswa dengan pemahaman cepat, dan satu lagi untuk siswa yang membutuhkan pendekatan lebih sederhana.

\textbf{3. Buat Kuis atau Latihan Evaluasi.}  
Gunakan LLM untuk menghasilkan soal evaluasi. Misalnya:
\begin{quote}
	\centering
	\texttt{“Buat 5 soal pilihan ganda dan 3 soal benar-salah tentang sifat-sifat cahaya untuk siswa kelas 6.”}
\end{quote}
Periksa apakah hasil kuis mencakup kunci jawaban dan penjelasan. Ubah redaksi bila ada kalimat yang kurang sesuai dengan gaya bahasa di kelas.

\textbf{4. Evaluasi Hasil dan Kelayakan Penggunaan.}  
Tinjau kembali hasil yang dibuat: apakah sesuai dengan kebutuhan peserta didik? Apakah tingkat kesulitan dan bahasa yang digunakan tepat? Apa yang perlu disesuaikan sebelum digunakan dalam proses pembelajaran?

\textbf{5. Simpan dan Bagikan dengan Rekan.}  
Jika hasilnya dirasa layak, simpan sebagai arsip materi ajar. Jika memungkinkan, bagikan kepada rekan pendidik untuk dijadikan referensi bersama atau kolaborasi materi.

Latihan ini bertujuan untuk mengasah keterampilan dalam mendesain materi pembelajaran yang responsif, efisien, dan relevan dengan perkembangan teknologi pengajaran berbasis AI.


\section*{Latihan Praktik: Mendesain Materi dan Kuis Adaptif}
\addcontentsline{toc}{section}{Latihan Praktik: Mendesain Materi dan Kuis Adaptif}
\begin{itemize}
	\item \textbf{Tujuan:} Membuat lembar kerja dan kuis yang sesuai dengan topik serta kebutuhan peserta didik.
	\item \textbf{Tugas:} Gunakan LLM untuk menghasilkan:
	\begin{itemize}
		\item Rencana pembelajaran berdurasi 1 jam untuk topik pilihan.
		\item Lima soal pilihan ganda beserta jawabannya.
		\item Lembar kerja yang disesuaikan untuk siswa dengan kebutuhan belajar khusus (misalnya: ringan teks, berpola gambar, atau bertingkat kesulitan).
	\end{itemize}
\end{itemize}
