\chapter{Etika dalam Penggunaan AI: Apa yang Bisa Salah?}

\section{Mengapa Etika Penting dalam Penggunaan AI di Pendidikan}
% Penjelasan tentang potensi risiko AI dan pentingnya sikap kritis terhadap hasil yang dihasilkan oleh LLM

\section{Bias dalam AI dan Dampaknya pada Pembelajaran}
% Bagaimana bias dalam data pelatihan dapat menyebabkan hasil yang tidak netral atau diskriminatif

\section{Risiko Misinformasi dan Fakta yang Salah}
% Penjelasan bahwa LLM bisa menghasilkan informasi yang tampak meyakinkan tetapi tidak akurat

\section{Ketergantungan Berlebih terhadap AI}
% Risiko menurunnya pemikiran kritis dan inisiatif siswa maupun guru jika terlalu mengandalkan AI

\section{Plagiarisme dan Keaslian Karya}
% Tantangan dalam membedakan karya orisinal dengan hasil AI serta bagaimana mengedukasi siswa tentang integritas akademik

\section{Privasi dan Keamanan Data}
% Bahaya menyisipkan data pribadi atau sensitif ke dalam sistem AI yang terhubung dengan internet

\section{Menggunakan AI secara Aman dan Bertanggung Jawab di Kelas}
% Praktik-praktik terbaik untuk memandu penggunaan AI secara bijak, aman, dan etis oleh guru dan siswa

\section{Latihan: Analisis Risiko dan Refleksi Etis}
% Aktivitas praktik berupa studi kasus, diskusi, dan pengambilan keputusan etis

\section*{Latihan Praktik: Etika dan Pengambilan Keputusan dalam Penggunaan AI}
\addcontentsline{toc}{section}{Latihan Praktik: Etika dan Pengambilan Keputusan dalam Penggunaan AI}
\begin{itemize}
	\item \textbf{Tujuan:} Mengenali potensi risiko penggunaan AI dan membangun sikap reflektif dalam pengambilan keputusan.
	\item \textbf{Tugas:}
	\begin{itemize}
		\item Analisis kasus: “Siswa mengumpulkan tugas hasil ChatGPT tanpa revisi” – apa langkah bijak yang dapat diambil?
		\item Buat daftar panduan penggunaan AI yang aman dan etis untuk siswa di kelas.
		\item Diskusikan batasan penggunaan AI untuk tugas, ujian, dan interaksi guru-siswa.
	\end{itemize}
\end{itemize}
