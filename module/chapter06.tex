\chapter{Etika dalam Penggunaan AI: Apa yang Bisa Salah?}

\section{Mengapa Etika Penting dalam Penggunaan AI di Pendidikan}

Kehadiran kecerdasan buatan (Artificial Intelligence/AI), khususnya model bahasa besar (LLM), telah membuka peluang besar untuk meningkatkan efisiensi dan kualitas dalam berbagai aspek pendidikan. Namun, seperti halnya teknologi lainnya, penggunaan AI dalam konteks pendidikan tidak lepas dari potensi risiko. Oleh karena itu, penting untuk membekali pendidik dengan kesadaran dan pemahaman etis agar penggunaan AI dapat dilakukan secara bertanggung jawab, adil, dan aman.

\textbf{Mengapa isu etika perlu mendapat perhatian khusus?}  
AI tidak memiliki kesadaran, penilaian moral, atau pemahaman kontekstual seperti manusia. Model seperti ChatGPT hanya menghasilkan respons berdasarkan pola data pelatihan yang sangat besar, tanpa benar-benar memahami kebenaran, nilai-nilai budaya, atau konsekuensi sosial. Hal ini membuat hasil yang diberikan AI bisa saja:
\begin{itemize}
	\item Menyesatkan atau tidak akurat
	\item Bias terhadap kelompok tertentu
	\item Terlalu bergantung pada data yang tidak lengkap atau tidak relevan
\end{itemize}

\textbf{Risiko penggunaan AI tanpa kesadaran etis antara lain:}
\begin{itemize}
	\item \textbf{Bias algoritma:} AI dapat memperkuat stereotip karena datanya berasal dari internet dan publikasi yang tidak selalu netral.
	\item \textbf{Plagiarisme:} Siswa bisa menyalahgunakan AI untuk menyalin tugas tanpa pemahaman atau proses belajar yang bermakna.
	\item \textbf{Misinformasi:} AI mungkin memberikan informasi yang tampak meyakinkan, tetapi ternyata tidak benar atau tidak relevan.
	\item \textbf{Ketergantungan berlebih:} Siswa atau guru bisa terlalu mengandalkan AI dan mengurangi daya pikir kritis atau kreativitasnya sendiri.
	\item \textbf{Privasi dan keamanan data:} Memasukkan informasi pribadi siswa ke dalam sistem AI publik dapat menimbulkan risiko kebocoran data.
\end{itemize}

\textbf{Mengapa guru perlu bersikap kritis terhadap hasil AI?}  
Guru perlu meninjau ulang setiap hasil yang diberikan AI, bukan hanya untuk memverifikasi kebenarannya, tetapi juga untuk memastikan bahwa hasil tersebut sesuai dengan konteks pembelajaran, karakter siswa, serta nilai-nilai budaya dan institusional. AI seharusnya dilihat sebagai \textit{alat bantu}, bukan pengganti proses pendidikan yang melibatkan relasi manusia, empati, dan pertimbangan pedagogis.

\textbf{Tanggung jawab etis dalam penggunaan AI di pendidikan meliputi:}
\begin{itemize}
	\item Menyampaikan secara jujur kepada siswa bahwa suatu materi dihasilkan oleh AI.
	\item Mengarahkan siswa untuk tetap berpikir kritis terhadap hasil AI.
	\item Menyunting dan menyesuaikan materi AI agar sesuai dengan nilai dan standar pendidikan.
	\item Melindungi identitas dan data pribadi siswa saat menggunakan layanan AI daring.
\end{itemize}

AI menghadirkan potensi luar biasa dalam mendukung pembelajaran, namun potensi ini hanya akan bermanfaat bila dikawal dengan prinsip-prinsip etika. Guru sebagai pendidik memiliki peran penting untuk memastikan bahwa pemanfaatan AI dilakukan dengan bijaksana—tidak hanya efisien, tetapi juga adil, aman, dan bermakna bagi pembelajaran jangka panjang. Pendidikan yang beretika di era digital adalah pendidikan yang sadar akan batas, tanggung jawab, dan nilai-nilai kemanusiaan.


\section{Bias dalam AI dan Dampaknya pada Pembelajaran}

Meskipun kecerdasan buatan (AI) seperti Large Language Models (LLM) dirancang untuk membantu menyelesaikan berbagai tugas secara efisien, penting untuk diingat bahwa sistem ini dilatih menggunakan data yang dikumpulkan dari dunia nyata. Data tersebut mencakup miliaran kata dari internet, buku, berita, forum, dan media sosial yang sering kali mencerminkan sudut pandang mayoritas, budaya dominan, atau bahkan asumsi dan stereotip yang tidak disadari. Akibatnya, AI tidak hanya belajar pola bahasa yang umum, tetapi juga bisa mewarisi \textbf{bias} yang terdapat dalam data tersebut.

\textbf{Apa itu bias dalam AI?}  
Bias dalam AI merujuk pada kecenderungan model untuk memberikan hasil atau prediksi yang tidak netral, memihak kelompok tertentu, atau mencerminkan prasangka yang tersembunyi dalam data pelatihan. Bias ini bisa bersifat:
\begin{itemize}
	\item \textbf{Bias budaya atau etnis}: misalnya AI lebih banyak menampilkan nama, referensi, atau gaya komunikasi dari budaya Barat.
	\item \textbf{Bias gender}: misalnya peran pekerjaan atau karakter diasosiasikan dengan jenis kelamin tertentu.
	\item \textbf{Bias bahasa atau regional}: model lebih akurat dalam menjawab dalam bahasa Inggris dibandingkan bahasa lain, atau lebih memahami konteks negara-negara tertentu.
	\item \textbf{Bias sosio-ekonomi}: informasi yang lebih mendukung kelompok tertentu, sementara kelompok marjinal terabaikan.
\end{itemize}

\textbf{Contoh dalam konteks pembelajaran:}
\begin{itemize}
	\item Ketika diminta memberikan tokoh inspiratif, AI lebih cenderung menyebut figur pria dari negara-negara maju.
	\item Ketika diminta membuat cerita anak, AI mungkin menggunakan nama-nama dari budaya tertentu secara dominan.
	\item Dalam membuat soal atau latihan, AI mungkin menyarankan konteks yang tidak relevan dengan latar belakang siswa Indonesia.
\end{itemize}

\textbf{Dampak terhadap pembelajaran:}
\begin{itemize}
	\item \textbf{Tidak inklusif}: siswa dari latar belakang minoritas bisa merasa tidak terwakili dalam materi yang dihasilkan oleh AI.
	\item \textbf{Penguatan stereotip}: jika tidak disadari, guru atau siswa bisa ikut melestarikan bias yang muncul dari AI.
	\item \textbf{Kehilangan konteks lokal}: materi belajar menjadi kurang relevan karena tidak sesuai dengan budaya, nilai, atau lingkungan siswa.
\end{itemize}

\textbf{Peran guru dalam mengelola bias AI:}
\begin{itemize}
	\item Meninjau dan menyunting hasil dari LLM agar lebih kontekstual dan adil.
	\item Menambahkan prompt yang memperjelas konteks lokal atau nilai-nilai yang ingin diangkat (misalnya: "sertakan tokoh dari Asia Tenggara").
	\item Menggunakan hasil AI sebagai titik awal diskusi kritis, bukan sebagai sumber kebenaran tunggal.
	\item Mendorong siswa untuk mengenali dan mempertanyakan bias, baik dalam AI maupun dalam sumber lain.
\end{itemize}

\textbf{Contoh prompt korektif:}

\begin{quote}
	\centering
	\texttt{"Buatkan daftar 5 tokoh inspiratif perempuan dari Asia Tenggara dalam bidang pendidikan."}
\end{quote}

AI bukanlah sistem yang netral secara otomatis. Ia merefleksikan dunia apa adanya, termasuk ketimpangan dan ketidakadilan yang ada di dalamnya. Oleh karena itu, penting bagi guru dan siswa untuk mendekati AI dengan sikap kritis dan etis. Ketika disadari dan dikelola dengan tepat, pembelajaran tidak hanya menjadi lebih adil dan inklusif, tetapi juga menjadi ruang refleksi yang memperkaya pemahaman terhadap keberagaman dunia nyata.


\section{Risiko Misinformasi dan Fakta yang Salah}

Salah satu tantangan terbesar dalam penggunaan Large Language Model (LLM) seperti ChatGPT dalam pendidikan adalah kemampuannya untuk menghasilkan informasi yang terdengar sangat meyakinkan, padahal belum tentu akurat. LLM dirancang untuk memprediksi kata atau kalimat berikutnya berdasarkan pola yang dipelajari dari data pelatihan. Ia tidak memiliki mekanisme internal untuk memverifikasi fakta seperti manusia yang merujuk pada sumber terpercaya. Akibatnya, model dapat menyampaikan jawaban yang salah atau menyesatkan—meskipun dikemas dengan gaya bahasa yang lugas, runtut, dan terlihat “pintar”.

\textbf{Apa itu misinformasi dalam konteks AI?}  
Misinformasi adalah informasi yang tidak benar atau tidak akurat, namun disampaikan seolah-olah benar. Dalam konteks AI, misinformasi bisa muncul karena:
\begin{itemize}
	\item Data pelatihan yang sudah usang atau tidak valid
	\item Kurangnya akses terhadap data yang diperbarui secara real-time
	\item Kesalahan dalam memahami konteks atau pertanyaan
	\item Permintaan pengguna yang ambigu atau kurang jelas
\end{itemize}

\textbf{Contoh misinformasi dari LLM:}
\begin{itemize}
	\item Memberikan tanggal yang salah dari peristiwa sejarah
	\item Mengutip penulis atau buku yang tidak ada
	\item Mengarang statistik atau hasil penelitian yang tidak pernah diterbitkan
	\item Memberikan definisi konsep yang keliru atau tidak sesuai standar ilmiah
\end{itemize}

\textbf{Mengapa ini berbahaya dalam pendidikan?}
\begin{itemize}
	\item \textbf{Siswa dapat menerima informasi yang keliru sebagai fakta}, jika tidak diajarkan untuk memverifikasi sumber.
	\item \textbf{Guru dapat menggunakan materi yang tidak akurat}, jika terlalu mengandalkan LLM tanpa proses pengecekan ulang.
	\item \textbf{Proses pembelajaran kehilangan validitas ilmiah}, yang seharusnya berbasis bukti dan referensi terpercaya.
\end{itemize}

\textbf{Contoh prompt yang rentan menyebabkan misinformasi:}
\begin{quote}
	\centering
	\texttt{"Apa 3 penemuan terpenting dari ilmuwan Indonesia yang memenangkan Nobel?"}
\end{quote}

Model bisa menjawab dengan menyebut nama ilmuwan Indonesia yang sebenarnya belum pernah menerima Nobel, karena ia "menebak" berdasarkan pola umum, bukan fakta aktual.

\textbf{Strategi untuk memitigasi risiko misinformasi:}
\begin{itemize}
	\item Gunakan LLM sebagai alat bantu awal, bukan sebagai satu-satunya sumber kebenaran.
	\item Selalu verifikasi jawaban dengan sumber primer atau referensi resmi.
	\item Ajarkan siswa untuk membandingkan informasi dari LLM dengan buku teks, jurnal, atau situs terpercaya.
	\item Tambahkan klausa penegasan pada prompt, seperti:
\end{itemize}

\begin{quote}
	\centering
	\texttt{"Berikan jawaban berdasarkan data yang diketahui publik hingga tahun 2023, dan sebutkan jika jawabannya bersifat asumsi atau tidak pasti."}
\end{quote}

\textbf{Peran guru sangat penting dalam hal ini:}  
Guru berperan sebagai penjamin kualitas informasi yang disampaikan di kelas. Dalam ekosistem yang didukung oleh AI, guru tidak hanya bertindak sebagai fasilitator pembelajaran, tetapi juga sebagai kurator pengetahuan—yang bertanggung jawab untuk memastikan bahwa informasi yang digunakan telah melalui proses validasi yang kritis.

Kekuatan LLM dalam menyampaikan informasi secara meyakinkan adalah pedang bermata dua. Di satu sisi, ia dapat menjelaskan konsep dengan sangat baik; namun di sisi lain, ia juga bisa menyesatkan jika tidak dikawal dengan kesadaran kritis. Oleh karena itu, kemampuan literasi informasi dan kebiasaan memverifikasi fakta menjadi kompetensi penting baik bagi guru

\section{Ketergantungan Berlebih terhadap AI}

Kemajuan teknologi kecerdasan buatan (Artificial Intelligence/AI), khususnya Large Language Model (LLM) seperti ChatGPT, telah membawa banyak manfaat dalam dunia pendidikan. AI mampu membantu menyusun materi pelajaran, merangkum teks, menilai tugas, bahkan merancang rencana pembelajaran secara otomatis. Namun, jika digunakan tanpa batas atau tanpa pemahaman kritis, AI berpotensi menimbulkan risiko serius—salah satunya adalah \textbf{ketergantungan berlebih} dari guru maupun siswa.

\textbf{Apa itu ketergantungan berlebih terhadap AI?}  
Ketergantungan berlebih terjadi ketika pengguna—baik guru maupun siswa—terlalu mengandalkan AI untuk menyelesaikan tugas-tugas intelektual dan kreatif, tanpa menyisakan ruang untuk berpikir kritis, inisiatif pribadi, atau proses reflektif yang mendalam. Alih-alih menjadi alat bantu, AI malah digunakan sebagai pengganti proses belajar yang seharusnya dijalani manusia.

\textbf{Contoh bentuk ketergantungan yang dapat muncul:}
\begin{itemize}
	\item Siswa menggunakan AI untuk mengerjakan semua tugas tanpa memahami isinya.
	\item Guru menyalin hasil LLM untuk membuat soal atau materi tanpa menyesuaikan konteks kelas.
	\item Siswa menjadi enggan membaca, menulis, atau berdiskusi karena semua informasi bisa "ditanyakan ke AI".
	\item Guru kehilangan semangat untuk bereksperimen dalam merancang pembelajaran karena merasa AI sudah "cukup membantu".
\end{itemize}

\textbf{Dampak negatif terhadap pembelajaran:}
\begin{itemize}
	\item \textbf{Menurunnya kemampuan berpikir kritis dan analitis}, karena siswa tidak terbiasa memproses informasi secara mandiri.
	\item \textbf{Hilangnya kreativitas dan orisinalitas}, baik dalam tulisan, proyek, maupun diskusi.
	\item \textbf{Kualitas pembelajaran yang generik dan kurang kontekstual}, karena terlalu bergantung pada output otomatis.
	\item \textbf{Ketimpangan pembelajaran}, karena siswa dengan literasi AI rendah menjadi semakin tertinggal.
\end{itemize}

\textbf{Mengapa ini menjadi perhatian penting?}  
Pendidikan bukan sekadar mencari jawaban, tetapi melatih proses berpikir, membangun pemahaman, dan menumbuhkan nilai-nilai. AI dapat mendukung proses ini, tetapi tidak bisa menggantikannya. AI tidak mampu merasakan, memahami emosi, atau mempertimbangkan konteks sosial yang penting dalam interaksi belajar. Jika terlalu sering digunakan tanpa pendampingan kritis, AI bisa mereduksi nilai-nilai kemanusiaan dalam proses pendidikan.

\textbf{Langkah pencegahan ketergantungan berlebih:}
\begin{itemize}
	\item Gunakan AI sebagai alat bantu, bukan sebagai solusi akhir.
	\item Dorong siswa untuk merefleksikan jawaban AI: \textit{“Apakah kamu setuju dengan jawaban ini? Mengapa atau mengapa tidak?”}
	\item Beri ruang bagi siswa untuk menciptakan, menulis, dan menyampaikan gagasan orisinal mereka.
	\item Ajak guru berdiskusi untuk memadukan hasil dari AI dengan strategi pedagogi yang kontekstual dan kreatif.
	\item Libatkan siswa dalam proses metakognitif: \textit{“Apa yang kamu pelajari setelah menggunakan AI?”}
\end{itemize}

AI memiliki potensi luar biasa untuk memperkuat proses pembelajaran, namun potensi ini harus digunakan dengan kesadaran penuh. Ketergantungan berlebih dapat menggerus daya nalar, rasa ingin tahu, dan tanggung jawab personal dalam belajar. Tugas utama pendidik adalah membimbing siswa agar tidak hanya “pandai menggunakan teknologi”, tetapi juga “cerdas dalam berpikir”—termasuk berpikir secara kritis terhadap hasil dari teknologi itu sendiri.

\section{Plagiarisme dan Keaslian Karya}

Di era digital yang semakin terhubung dan didukung oleh kecerdasan buatan (AI), tantangan terhadap keaslian karya akademik semakin kompleks. Dengan kehadiran model bahasa besar (LLM) seperti ChatGPT, siswa kini dapat menghasilkan teks yang tampak rapi, logis, dan akademis hanya dengan satu perintah. Meskipun AI dapat menjadi alat bantu yang luar biasa dalam proses belajar, penggunaannya juga membuka peluang terjadinya \textbf{plagiarisme berbasis teknologi}, yakni praktik menyerahkan karya yang bukan hasil pemikiran sendiri tanpa pengakuan yang layak.

\textbf{Apa yang dimaksud dengan plagiarisme di era AI?}  
Plagiarisme tidak lagi terbatas pada menyalin teks dari buku atau internet secara langsung, tetapi juga meliputi:
\begin{itemize}
	\item Menyerahkan hasil dari AI sebagai karya pribadi tanpa revisi atau refleksi.
	\item Menggunakan AI untuk menjawab soal esai tanpa pemahaman substansi.
	\item Menyisipkan kalimat atau paragraf dari hasil AI tanpa mencantumkan bahwa itu dibantu oleh mesin.
\end{itemize}

\textbf{Tantangan utama:}
\begin{itemize}
	\item \textbf{Sulit dibedakan}: Teks hasil AI sering kali terdengar alami dan orisinal, sehingga sulit untuk dideteksi secara kasat mata.
	\item \textbf{Tidak terdeteksi oleh alat cek plagiarisme biasa}: Banyak hasil AI tidak terdaftar dalam database publik, sehingga tidak muncul dalam hasil cek.
	\item \textbf{Kurangnya pemahaman siswa}: Sebagian siswa tidak sadar bahwa menyerahkan hasil AI secara utuh tanpa pengakuan adalah bentuk ketidakjujuran akademik.
\end{itemize}

\textbf{Mengapa keaslian karya penting?}  
Tujuan utama dari tugas akademik bukan hanya menyelesaikan soal, melainkan membentuk:
\begin{itemize}
	\item Kemampuan berpikir kritis dan analitis
	\item Daya nalar dan kreativitas pribadi
	\item Tanggung jawab intelektual dan integritas
\end{itemize}

Ketika siswa tidak benar-benar terlibat dalam proses berpikir, mereka kehilangan kesempatan untuk tumbuh sebagai pembelajar sejati. Pendidikan menjadi dangkal jika hanya berfokus pada “hasil” dan bukan “proses”.

\textbf{Cara mengedukasi siswa tentang integritas akademik:}
\begin{itemize}
	\item Jelaskan dengan jelas apa itu plagiarisme, termasuk penggunaan AI tanpa atribusi.
	\item Ajak siswa berdiskusi tentang perbedaan antara \textit{mendapat bantuan} dan \textit{menyerahkan hasil orang lain}.
	\item Terapkan refleksi mandiri sebagai bagian dari tugas: \textit{“Bagian mana yang kamu buat sendiri, dan bagian mana yang dibantu AI?”}
	\item Tunjukkan contoh prompt yang etis, misalnya:
\end{itemize}

\begin{quote}
	\centering
	\texttt{"Bantu saya membuat kerangka esai tentang dampak teknologi, tetapi saya akan menulis isi lengkapnya sendiri."}
\end{quote}

\textbf{Peran guru dan sekolah:}
\begin{itemize}
	\item Memberikan ruang diskusi terbuka tentang etika digital dan literasi AI.
	\item Menyediakan panduan atau kebijakan jelas tentang penggunaan AI dalam tugas sekolah.
	\item Menilai tidak hanya hasil akhir, tetapi juga proses pembuatan karya.
	\item Menggunakan AI detection tools dengan bijak, bukan sebagai alat penghukuman semata.
\end{itemize}
  
Plagiarisme bukan sekadar tindakan menyalin, tetapi kehilangan kesempatan untuk belajar dan berkembang. AI tidak seharusnya dihindari, tetapi perlu digunakan dengan kesadaran penuh dan integritas. Tugas pendidik adalah membekali siswa dengan keterampilan dan nilai untuk menggunakan teknologi secara bijak, bertanggung jawab, dan tetap menjunjung tinggi keaslian sebagai bentuk penghargaan terhadap proses belajar.


\section{Privasi dan Keamanan Data}

Di tengah semakin meluasnya pemanfaatan teknologi kecerdasan buatan (AI) dalam pendidikan, muncul tantangan besar terkait \textbf{privasi dan keamanan data}. Banyak layanan AI, termasuk model bahasa besar (Large Language Model/LLM), diakses melalui internet dan berbasis cloud. Ini berarti semua masukan (input) yang diketikkan ke dalam sistem—termasuk informasi pribadi, catatan siswa, atau data institusi—dapat tersimpan, dianalisis, atau diproses oleh server pihak ketiga. Tanpa kesadaran akan risiko ini, guru maupun siswa bisa secara tidak sengaja membahayakan kerahasiaan data pendidikan.

\textbf{Mengapa privasi penting dalam penggunaan AI?}  
Dalam konteks pendidikan, data yang diproses sering kali mengandung informasi yang bersifat sensitif, seperti:
\begin{itemize}
	\item Nama lengkap dan identitas siswa
	\item Data kesehatan atau kebutuhan khusus
	\item Catatan perkembangan belajar dan penilaian
	\item Masalah perilaku atau psikososial
	\item Informasi internal sekolah atau lembaga
\end{itemize}

Jika informasi-informasi tersebut dimasukkan secara utuh ke dalam sistem AI publik (seperti ChatGPT versi online), maka data tersebut \textit{berpotensi} digunakan untuk pelatihan model lebih lanjut, disimpan oleh penyedia layanan, atau—pada kasus ekstrem—terekspos melalui pelanggaran keamanan.

\textbf{Contoh kesalahan umum:}
\begin{itemize}
	\item Guru meminta AI menyusun laporan siswa dengan memasukkan nama lengkap, prestasi, dan masalah belajar secara langsung.
	\item Siswa menyalin transkrip konseling ke AI untuk “minta saran”.
	\item Pegawai sekolah menggunakan AI untuk menulis surat terkait data keuangan atau administratif yang bersifat rahasia.
\end{itemize}

\textbf{Risiko yang dapat timbul:}
\begin{itemize}
	\item \textbf{Kebocoran data pribadi siswa}
	\item \textbf{Pelanggaran hukum perlindungan data (misalnya UU PDP atau GDPR)}
	\item \textbf{Disalahgunakan oleh pihak tidak bertanggung jawab}
	\item \textbf{Kehilangan kepercayaan antara sekolah dan orang tua}
\end{itemize}

\textbf{Prinsip dasar menjaga privasi saat menggunakan AI:}
\begin{itemize}
	\item Jangan pernah memasukkan informasi identitas yang lengkap ke dalam layanan AI publik.
	\item Gunakan nama samaran, inisial, atau deskripsi umum jika perlu membuat prompt berbasis kasus nyata.
	\item Pilih layanan AI yang menawarkan mode lokal, privat, atau tidak menyimpan data.
	\item Baca dan pahami kebijakan privasi platform AI sebelum digunakan dalam kegiatan pembelajaran.
	\item Jika memungkinkan, gunakan sistem AI internal sekolah yang dikendalikan secara lokal.
\end{itemize}

\textbf{Contoh prompt yang lebih aman:}

\begin{quote}
	\centering
	\texttt{"Buat laporan perkembangan siswa berdasarkan karakter berikut: siswa aktif, memiliki kesulitan membaca, sangat baik dalam kerja kelompok. Gunakan nama samaran 'Siswa A'."}
\end{quote}

\textbf{Peran pendidik dan institusi:}
\begin{itemize}
	\item Memberikan pelatihan kepada guru dan siswa tentang etika digital dan keamanan data.
	\item Menyusun kebijakan internal sekolah tentang batasan penggunaan AI.
	\item Menyediakan alternatif layanan AI yang ramah privasi.
\end{itemize}

AI dapat membantu mempercepat banyak aspek kerja pendidik, tetapi tidak boleh digunakan secara sembarangan—terutama jika menyangkut data pribadi. Menjaga privasi bukan hanya soal etika, tetapi juga soal hukum dan keamanan. Dengan sikap hati-hati dan kebijakan yang jelas, teknologi AI dapat dimanfaatkan secara bertanggung jawab dan tetap melindungi hak-hak peserta didik.

\section{Menggunakan AI secara Aman dan Bertanggung Jawab di Kelas}

Penggunaan AI dalam pembelajaran membawa banyak manfaat, mulai dari membantu menyusun materi ajar, memberikan umpan balik otomatis, hingga mendukung siswa dalam menulis, menganalisis, dan menyederhanakan informasi. Namun, seperti halnya teknologi lainnya, AI harus digunakan secara \textbf{bijak, aman, dan bertanggung jawab}—baik oleh guru maupun siswa. Tanpa panduan yang tepat, penggunaan AI dapat mengarah pada ketergantungan, penyebaran misinformasi, pelanggaran privasi, bahkan pelanggaran etika akademik.

\textbf{Mengapa pedoman penggunaan AI di kelas penting?}  
AI bukan sekadar alat bantu teknis, tetapi juga bagian dari proses pembelajaran. Penggunaannya harus mendukung kompetensi berpikir kritis, literasi digital, dan pembentukan karakter siswa. Tanpa regulasi dan edukasi yang baik, teknologi yang seharusnya memperkuat pembelajaran justru bisa melemahkan esensi belajar itu sendiri.

\textbf{Prinsip-prinsip penggunaan AI yang aman dan bertanggung jawab:}
\begin{itemize}
	\item \textbf{Transparansi}: Jelaskan kepada siswa saat materi atau penilaian dibuat dengan bantuan AI.
	\item \textbf{Refleksi}: Dorong siswa untuk mengevaluasi dan mempertanyakan hasil dari AI, bukan menerimanya secara mentah.
	\item \textbf{Konfirmasi}: Ajarkan siswa untuk selalu membandingkan informasi dari AI dengan sumber terpercaya.
	\item \textbf{Privasi}: Jangan memasukkan data pribadi, nama lengkap, atau informasi sensitif ke dalam sistem AI terbuka.
	\item \textbf{Orisinalitas}: Gunakan AI sebagai alat bantu berpikir, bukan untuk menggantikan proses berkarya atau berargumentasi.
\end{itemize}

\textbf{Contoh aktivitas yang mendukung penggunaan AI yang etis:}
\begin{itemize}
	\item Menganalisis dua versi teks: hasil buatan manusia vs hasil AI, dan membahas kelebihan/kekurangannya.
	\item Memberikan tugas yang menggabungkan bantuan AI dengan refleksi pribadi siswa.
	\item Mengajak siswa menyusun “Pedoman Penggunaan AI di Kelas” versi mereka sendiri.
\end{itemize}

\textbf{Tugas guru dalam memandu penggunaan AI:}
\begin{itemize}
	\item Menyediakan konteks: bantu siswa memahami bahwa AI tidak "tahu segalanya", tapi hanya memprediksi berdasarkan data.
	\item Menilai proses, bukan hanya produk akhir: beri apresiasi terhadap usaha berpikir dan refleksi.
	\item Memfasilitasi diskusi tentang nilai, etika, dan konsekuensi dalam penggunaan teknologi.
\end{itemize}

\textbf{Contoh pernyataan yang dapat digunakan guru saat mengajar:}
\begin{quote}
	“Silakan gunakan ChatGPT untuk membantu menjelaskan konsep ini, tapi pastikan kamu memahami isinya dan bisa menjelaskan dengan kata-katamu sendiri.”
\end{quote}

\textbf{Checklist singkat untuk penggunaan AI di kelas:}
\begin{itemize}
	\item Sudahkah saya memahami hasil yang diberikan AI?
	\item Apakah saya sudah mengecek sumber lain?
	\item Apakah saya menyebutkan bahwa saya menggunakan bantuan AI?
	\item Apakah informasi yang saya masukkan tidak bersifat pribadi?
	\item Apakah saya sudah menambahkan ide atau refleksi saya sendiri?
\end{itemize}
  
AI dapat menjadi mitra cerdas dalam proses pembelajaran, tetapi hanya jika digunakan dengan tanggung jawab dan kesadaran etis. Guru berperan penting sebagai pendamping dan pengarah dalam memfasilitasi pemanfaatan teknologi ini. Dengan membangun budaya berpikir kritis dan literasi digital, kita dapat memastikan bahwa AI mendukung pendidikan yang bermakna, bukan menggantikannya.


\section{Latihan Analisis Risiko dan Refleksi Etis}

Agar pemahaman mengenai penggunaan AI dalam pendidikan tidak hanya bersifat teoretis, peserta pelatihan perlu dilibatkan dalam aktivitas praktik yang mendorong analisis kritis, refleksi nilai, dan pengambilan keputusan secara etis. Latihan ini bertujuan untuk mengembangkan sensitivitas peserta terhadap potensi risiko yang mungkin muncul dari penggunaan AI, sekaligus memperkuat pemahaman tentang bagaimana bersikap bijak dan bertanggung jawab dalam mengintegrasikan teknologi ke dalam pembelajaran.

\textbf{Tujuan latihan:}
\begin{itemize}
	\item Mengidentifikasi risiko etis dan teknis dalam penggunaan AI di kelas
	\item Menganalisis studi kasus nyata atau simulatif secara kritis
	\item Membiasakan diri untuk mempertimbangkan aspek keamanan, keadilan, dan integritas akademik
	\item Merumuskan prinsip atau pedoman praktis untuk penggunaan AI yang bertanggung jawab
\end{itemize}

\textbf{Langkah-langkah aktivitas:}

\textbf{1. Studi Kasus: Identifikasi Risiko}
Berikan beberapa studi kasus singkat yang menggambarkan situasi dilematis dalam penggunaan AI. Contohnya:

\begin{quote}
	\textbf{Kasus A:} Seorang siswa menyerahkan esai yang ditulis sepenuhnya menggunakan ChatGPT tanpa menyunting atau menyebutkan penggunaan AI. Guru merasa isi esainya sangat rapi dan mencurigakan.
\end{quote}

\begin{quote}
	\textbf{Kasus B:} Guru memasukkan catatan perkembangan siswa ke dalam AI berbasis cloud untuk membuat laporan, tanpa menyadari bahwa data tersebut mengandung informasi pribadi dan sensitif.
\end{quote}

\textbf{2. Diskusi Kelompok: Analisis dan Pendapat}
Peserta dibagi ke dalam kelompok kecil. Setiap kelompok diminta untuk mendiskusikan:
\begin{itemize}
	\item Apa potensi risiko yang muncul dari kasus tersebut?
	\item Apakah ada pelanggaran terhadap prinsip etika atau privasi?
	\item Apa tindakan alternatif yang lebih tepat?
	\item Jika Anda adalah guru dalam kasus tersebut, apa keputusan yang Anda ambil?
\end{itemize}

\textbf{3. Presentasi dan Debrief}
Setiap kelompok mempresentasikan hasil diskusinya. Fasilitator dapat memancing refleksi lanjutan dengan pertanyaan seperti:
\begin{itemize}
	\item Bagaimana membedakan bantuan AI dengan bentuk kecurangan?
	\item Apa batas yang perlu disepakati dalam penggunaan AI di kelas?
	\item Apa tanggung jawab guru terhadap hasil yang dihasilkan oleh teknologi?
\end{itemize}

\textbf{4. Refleksi Individu Tertulis}
Minta peserta menuliskan refleksi pribadi:  
\textit{“Apa satu hal baru yang saya sadari tentang penggunaan AI secara etis di kelas?”}  
\textit{“Apa yang akan saya ubah dalam praktik saya ke depan setelah latihan ini?”}

\textbf{5. Merancang Pedoman Mini}
Sebagai tugas akhir, peserta diminta menyusun satu paragraf pedoman atau komitmen pribadi tentang bagaimana mereka akan menggunakan AI secara etis dan bertanggung jawab dalam konteks pembelajaran.

\textbf{Contoh hasil pedoman peserta:}
\begin{quote}
	"Saya akan menggunakan AI sebagai alat bantu pembelajaran, bukan sebagai pengganti proses berpikir siswa. Saya akan selalu meninjau dan menyesuaikan hasil AI, serta menjaga kerahasiaan data siswa dalam setiap interaksi digital."
\end{quote}

\textbf{Manfaat latihan ini:}
\begin{itemize}
	\item Menumbuhkan kesadaran kritis terhadap teknologi
	\item Membantu peserta mengembangkan standar etika pribadi
	\item Menyiapkan guru untuk menjadi role model dalam literasi digital dan tanggung jawab teknologi
\end{itemize}

Analisis risiko dan refleksi etis adalah langkah penting dalam memastikan AI digunakan dengan cara yang memperkuat nilai-nilai pendidikan, bukan hanya sebagai alat otomatisasi. Melalui studi kasus dan diskusi, peserta tidak hanya belajar tentang risiko teknologi, tetapi juga mengembangkan kompas moral untuk menghadapinya secara bijak.

\section*{Latihan Praktik: Etika dan Pengambilan Keputusan dalam Penggunaan AI}
\addcontentsline{toc}{section}{Latihan Praktik: Etika dan Pengambilan Keputusan dalam Penggunaan AI}
\begin{itemize}
	\item \textbf{Tujuan:} Mengenali potensi risiko penggunaan AI dan membangun sikap reflektif dalam pengambilan keputusan.
	\item \textbf{Tugas:}
	\begin{itemize}
		\item Analisis kasus: "Siswa mengumpulkan tugas hasil ChatGPT tanpa revisi" – apa langkah bijak yang dapat diambil?
		\item Buat daftar panduan penggunaan AI yang aman dan etis untuk siswa di kelas.
		\item Diskusikan batasan penggunaan AI untuk tugas, ujian, dan interaksi guru-siswa.
	\end{itemize}
\end{itemize}
